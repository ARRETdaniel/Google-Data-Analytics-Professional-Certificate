%% abtex2-modelo-trabalho-academico.tex, v-1.9.5 laurocesar
%% Copyright 2012-2015 by abnTeX2 group at http://www.abntex.net.br/
%%
%% This work may be distributed and/or modified under the
%% conditions of the LaTeX Project Public License, either version 1.3
%% of this license or (at your option) any later version.
%% The latest version of this license is in
%%   http://www.latex-project.org/lppl.txt
%% and version 1.3 or later is part of all distributions of LaTeX
%% version 2005/12/01 or later.
%%
%% This work has the LPPL maintenance status `maintained'.
%%
%% The Current Maintainer of this work is the abnTeX2 team, led
%% by Lauro César Araujo. Further information are available on
%% http://www.abntex.net.br/
%%
%% This work consists of the files abntex2-modelo-trabalho-academico.tex,
%% abntex2-modelo-include-comandos and abntex2-modelo-references.bib
%%

% ------------------------------------------------------------------------
% ------------------------------------------------------------------------
% abnTeX2: Modelo de Trabalho Academico (tese de doutorado, dissertacao de
% mestrado e trabalhos monograficos em geral) em conformidade com
% ABNT NBR 14724:2011: Informacao e documentacao - Trabalhos academicos -
% Apresentacao
% ------------------------------------------------------------------------
% ------------------------------------------------------------------------

\documentclass[
	% -- opções da classe memoir --
	10pt,				% tamanho da fonte
	%openright,			% capítulos começam em pág ímpar (insere página vazia caso preciso)
	openany, % a chapter can start on any page, then many classes support option openany, e.g.:
	%oneside, % With twoside layout (default for class book) chapters start at odd numbered pages and sometimes LaTeX needs to insert a page to ensure this.
	%twoside,			% para impressão em verso e anverso. Oposto a oneside
	a4paper,			% tamanho do papel.
	% -- opções da classe abntex2 --
	%chapter=TITLE,		% títulos de capítulos convertidos em letras maiúsculas
	%section=TITLE,		% títulos de seções convertidos em letras maiúsculas
	%subsection=TITLE,	% títulos de subseções convertidos em letras maiúsculas
	%subsubsection=TITLE,% títulos de subsubseções convertidos em letras maiúsculas
	% -- opções do pacote babel --
	english,			% idioma adicional para hifenização
	french,				% idioma adicional para hifenização
	spanish,			% idioma adicional para hifenização
	brazil				% o último idioma é o principal do documento
	]{abntex2}

% ---
% Pacotes básicos
% ---
\usepackage{lmodern}			% Usa a fonte Latin Modern
\usepackage[T1]{fontenc}		% Selecao de codigos de fonte.
\usepackage[utf8]{inputenc}		% Codificacao do documento (conversão automática dos acentos)
\usepackage{lastpage}			% Usado pela Ficha catalográfica
\usepackage{indentfirst}		% Indenta o primeiro parágrafo de cada seção.
\usepackage{color}				% Controle das cores
\usepackage{graphicx}			% Inclusão de gráficos
\usepackage{microtype} 			% para melhorias de justificação
\usepackage{booktabs}
\usepackage{graphicx}
\usepackage[table,xcdraw]{xcolor}
\usepackage{float}
\usepackage{listings}
\usepackage{ragged2e}


% --- https://www.overleaf.com/learn/latex/Code_listing



\usepackage{xcolor}
\renewcommand\lstlistingname{Lista de código}
\renewcommand\lstlistlistingname{Lista de trechos de código}

% https://tex.stackexchange.com/questions/111580/removing-an-unwanted-page-between-two-chapters
\let\cleardoublepage\clearpage % Unwanted one-page gap between two chapters can be eliminated using the syntax




\definecolor{codegreen}{rgb}{0,0.6,0}
\definecolor{codegray}{rgb}{0.5,0.5,0.5}
\definecolor{codepurple}{rgb}{0.58,0,0.82}
\definecolor{backcolour}{rgb}{0.95,0.95,0.92}

\lstdefinestyle{mystyle}{
    backgroundcolor=\color{backcolour},
    commentstyle=\color{codegreen},
    keywordstyle=\color{magenta},
    numberstyle=\tiny\color{codegray},
    stringstyle=\color{codepurple},
    basicstyle=\ttfamily\footnotesize,
    breakatwhitespace=false,
    breaklines=true,
    captionpos=b,
    keepspaces=true,
    numbers=left,
    numbersep=5pt,
    showspaces=false,
    showstringspaces=false,
    showtabs=false,
    tabsize=2
}

\lstset{style=mystyle}





% ---
% Pacotes adicionais, usados apenas no âmbito do Modelo Canônico do abnteX2
% ---
\usepackage{lipsum}				% para geração de dummy text
% ---

% ---
% Pacotes de citações
% ---
\usepackage[brazilian,hyperpageref]{backref}	 % Paginas com as citações na bibl
\usepackage[alf]{abntex2cite}	% Citações padrão ABNT

% ---
% First pip install pygments

% CONFIGURAÇÕES DE PACOTES
% ---

% ---
% Configurações do pacote backref
% Usado sem a opção hyperpageref de backref
\renewcommand{\backrefpagesname}{Citado na(s) página(s):~}
% Texto padrão antes do número das páginas
\renewcommand{\backref}{}
% Define os textos da citação
\renewcommand*{\backrefalt}[4]{
	\ifcase #1 %
		Nenhuma citação no texto.%
	\or
		Citado na página #2.%
	\else
		Citado #1 vezes nas páginas #2.%
	\fi}%
% ---
\usepackage[breakable]{tcolorbox}
\usepackage{parskip} % Stop auto-indenting (to mimic markdown behaviour)


% Basic figure setup, for now with no caption control since it's done
% automatically by Pandoc (which extracts ![](path) syntax from Markdown).
\usepackage{graphicx}
% Maintain compatibility with old templates. Remove in nbconvert 6.0
\let\Oldincludegraphics\includegraphics
% Ensure that by default, figures have no caption (until we provide a
% proper Figure object with a Caption API and a way to capture that
% in the conversion process - todo).
\usepackage{caption}
\DeclareCaptionFormat{nocaption}{}
\captionsetup{format=nocaption,aboveskip=0pt,belowskip=0pt}

\usepackage{float}
\floatplacement{figure}{H} % forces figures to be placed at the correct location
\usepackage{xcolor} % Allow colors to be defined
\usepackage{enumerate} % Needed for markdown enumerations to work
\usepackage{geometry} % Used to adjust the document margins
\usepackage{amsmath} % Equations
\usepackage{amssymb} % Equations
\usepackage{textcomp} % defines textquotesingle
% Hack from http://tex.stackexchange.com/a/47451/13684:
\AtBeginDocument{%
		\def\PYZsq{\textquotesingle}% Upright quotes in Pygmentized code
}
\usepackage{upquote} % Upright quotes for verbatim code
\usepackage{eurosym} % defines \euro

\usepackage{iftex}
\ifPDFTeX
		\usepackage[T1]{fontenc}
		\IfFileExists{alphabeta.sty}{
					\usepackage{alphabeta}
			}{
					\usepackage[mathletters]{ucs}
					\usepackage[utf8x]{inputenc}
			}
\else
		\usepackage{fontspec}
		\usepackage{unicode-math}
\fi

\usepackage{fancyvrb} % verbatim replacement that allows latex
\usepackage{grffile} % extends the file name processing of package graphics
										 % to support a larger range
\makeatletter % fix for old versions of grffile with XeLaTeX
\@ifpackagelater{grffile}{2019/11/01}
{
	% Do nothing on new versions
}
{
	\def\Gread@@xetex#1{%
		\IfFileExists{"\Gin@base".bb}%
		{\Gread@eps{\Gin@base.bb}}%
		{\Gread@@xetex@aux#1}%
	}
}
\makeatother
\usepackage[Export]{adjustbox} % Used to constrain images to a maximum size
\adjustboxset{max size={0.9\linewidth}{0.9\paperheight}}

% The hyperref package gives us a pdf with properly built
% internal navigation ('pdf bookmarks' for the table of contents,
% internal cross-reference links, web links for URLs, etc.)
\usepackage{hyperref}
% The default LaTeX title has an obnoxious amount of whitespace. By default,
% titling removes some of it. It also provides customization options.
\usepackage{titling}
\usepackage{longtable} % longtable support required by pandoc >1.10
\usepackage{booktabs}  % table support for pandoc > 1.12.2
\usepackage{array}     % table support for pandoc >= 2.11.3
\usepackage{calc}      % table minipage width calculation for pandoc >= 2.11.1
\usepackage[inline]{enumitem} % IRkernel/repr support (it uses the enumerate* environment)
%\usepackage[normalem]{ulem} % ulem is needed to support strikethroughs (\sout)
														% normalem makes italics be italics, not underlines
%\usepackage{mathrsfs}



% Colors for the hyperref package
\definecolor{urlcolor}{rgb}{0,.145,.698}
\definecolor{linkcolor}{rgb}{.71,0.21,0.01}
\definecolor{citecolor}{rgb}{.12,.54,.11}

% ANSI colors
\definecolor{ansi-black}{HTML}{3E424D}
\definecolor{ansi-black-intense}{HTML}{282C36}
\definecolor{ansi-red}{HTML}{E75C58}
\definecolor{ansi-red-intense}{HTML}{B22B31}
\definecolor{ansi-green}{HTML}{00A250}
\definecolor{ansi-green-intense}{HTML}{007427}
\definecolor{ansi-yellow}{HTML}{DDB62B}
\definecolor{ansi-yellow-intense}{HTML}{B27D12}
\definecolor{ansi-blue}{HTML}{208FFB}
\definecolor{ansi-blue-intense}{HTML}{0065CA}
\definecolor{ansi-magenta}{HTML}{D160C4}
\definecolor{ansi-magenta-intense}{HTML}{A03196}
\definecolor{ansi-cyan}{HTML}{60C6C8}
\definecolor{ansi-cyan-intense}{HTML}{258F8F}
\definecolor{ansi-white}{HTML}{C5C1B4}
\definecolor{ansi-white-intense}{HTML}{A1A6B2}
\definecolor{ansi-default-inverse-fg}{HTML}{FFFFFF}
\definecolor{ansi-default-inverse-bg}{HTML}{000000}

% common color for the border for error outputs.
\definecolor{outerrorbackground}{HTML}{FFDFDF}

% commands and environments needed by pandoc snippets
% extracted from the output of `pandoc -s`
\providecommand{\tightlist}{%
	\setlength{\itemsep}{0pt}\setlength{\parskip}{0pt}}
\DefineVerbatimEnvironment{Highlighting}{Verbatim}{commandchars=\\\{\}}
% Add ',fontsize=\small' for more characters per line
\newenvironment{Shaded}{}{}
\newcommand{\KeywordTok}[1]{\textcolor[rgb]{0.00,0.44,0.13}{\textbf{{#1}}}}
\newcommand{\DataTypeTok}[1]{\textcolor[rgb]{0.56,0.13,0.00}{{#1}}}
\newcommand{\DecValTok}[1]{\textcolor[rgb]{0.25,0.63,0.44}{{#1}}}
\newcommand{\BaseNTok}[1]{\textcolor[rgb]{0.25,0.63,0.44}{{#1}}}
\newcommand{\FloatTok}[1]{\textcolor[rgb]{0.25,0.63,0.44}{{#1}}}
\newcommand{\CharTok}[1]{\textcolor[rgb]{0.25,0.44,0.63}{{#1}}}
\newcommand{\StringTok}[1]{\textcolor[rgb]{0.25,0.44,0.63}{{#1}}}
\newcommand{\CommentTok}[1]{\textcolor[rgb]{0.38,0.63,0.69}{\textit{{#1}}}}
\newcommand{\OtherTok}[1]{\textcolor[rgb]{0.00,0.44,0.13}{{#1}}}
\newcommand{\AlertTok}[1]{\textcolor[rgb]{1.00,0.00,0.00}{\textbf{{#1}}}}
\newcommand{\FunctionTok}[1]{\textcolor[rgb]{0.02,0.16,0.49}{{#1}}}
\newcommand{\RegionMarkerTok}[1]{{#1}}
\newcommand{\ErrorTok}[1]{\textcolor[rgb]{1.00,0.00,0.00}{\textbf{{#1}}}}
\newcommand{\NormalTok}[1]{{#1}}

% Additional commands for more recent versions of Pandoc
\newcommand{\ConstantTok}[1]{\textcolor[rgb]{0.53,0.00,0.00}{{#1}}}
\newcommand{\SpecialCharTok}[1]{\textcolor[rgb]{0.25,0.44,0.63}{{#1}}}
\newcommand{\VerbatimStringTok}[1]{\textcolor[rgb]{0.25,0.44,0.63}{{#1}}}
\newcommand{\SpecialStringTok}[1]{\textcolor[rgb]{0.73,0.40,0.53}{{#1}}}
\newcommand{\ImportTok}[1]{{#1}}
\newcommand{\DocumentationTok}[1]{\textcolor[rgb]{0.73,0.13,0.13}{\textit{{#1}}}}
\newcommand{\AnnotationTok}[1]{\textcolor[rgb]{0.38,0.63,0.69}{\textbf{\textit{{#1}}}}}
\newcommand{\CommentVarTok}[1]{\textcolor[rgb]{0.38,0.63,0.69}{\textbf{\textit{{#1}}}}}
\newcommand{\VariableTok}[1]{\textcolor[rgb]{0.10,0.09,0.49}{{#1}}}
\newcommand{\ControlFlowTok}[1]{\textcolor[rgb]{0.00,0.44,0.13}{\textbf{{#1}}}}
\newcommand{\OperatorTok}[1]{\textcolor[rgb]{0.40,0.40,0.40}{{#1}}}
\newcommand{\BuiltInTok}[1]{{#1}}
\newcommand{\ExtensionTok}[1]{{#1}}
\newcommand{\PreprocessorTok}[1]{\textcolor[rgb]{0.74,0.48,0.00}{{#1}}}
\newcommand{\AttributeTok}[1]{\textcolor[rgb]{0.49,0.56,0.16}{{#1}}}
\newcommand{\InformationTok}[1]{\textcolor[rgb]{0.38,0.63,0.69}{\textbf{\textit{{#1}}}}}
\newcommand{\WarningTok}[1]{\textcolor[rgb]{0.38,0.63,0.69}{\textbf{\textit{{#1}}}}}


% Define a nice break command that doesn't care if a line doesn't already
% exist.
\def\br{\hspace*{\fill} \\* }
% Math Jax compatibility definitions
\def\gt{>}
\def\lt{<}
\let\Oldtex\TeX
\let\Oldlatex\LaTeX
\renewcommand{\TeX}{\textrm{\Oldtex}}
\renewcommand{\LaTeX}{\textrm{\Oldlatex}}
% Document parameters
% Document title
\title{finalproject}





% Pygments definitions
\makeatletter
\def\PY@reset{\let\PY@it=\relax \let\PY@bf=\relax%
\let\PY@ul=\relax \let\PY@tc=\relax%
\let\PY@bc=\relax \let\PY@ff=\relax}
\def\PY@tok#1{\csname PY@tok@#1\endcsname}
\def\PY@toks#1+{\ifx\relax#1\empty\else%
\PY@tok{#1}\expandafter\PY@toks\fi}
\def\PY@do#1{\PY@bc{\PY@tc{\PY@ul{%
\PY@it{\PY@bf{\PY@ff{#1}}}}}}}
\def\PY#1#2{\PY@reset\PY@toks#1+\relax+\PY@do{#2}}

\@namedef{PY@tok@w}{\def\PY@tc##1{\textcolor[rgb]{0.73,0.73,0.73}{##1}}}
\@namedef{PY@tok@c}{\let\PY@it=\textit\def\PY@tc##1{\textcolor[rgb]{0.24,0.48,0.48}{##1}}}
\@namedef{PY@tok@cp}{\def\PY@tc##1{\textcolor[rgb]{0.61,0.40,0.00}{##1}}}
\@namedef{PY@tok@k}{\let\PY@bf=\textbf\def\PY@tc##1{\textcolor[rgb]{0.00,0.50,0.00}{##1}}}
\@namedef{PY@tok@kp}{\def\PY@tc##1{\textcolor[rgb]{0.00,0.50,0.00}{##1}}}
\@namedef{PY@tok@kt}{\def\PY@tc##1{\textcolor[rgb]{0.69,0.00,0.25}{##1}}}
\@namedef{PY@tok@o}{\def\PY@tc##1{\textcolor[rgb]{0.40,0.40,0.40}{##1}}}
\@namedef{PY@tok@ow}{\let\PY@bf=\textbf\def\PY@tc##1{\textcolor[rgb]{0.67,0.13,1.00}{##1}}}
\@namedef{PY@tok@nb}{\def\PY@tc##1{\textcolor[rgb]{0.00,0.50,0.00}{##1}}}
\@namedef{PY@tok@nf}{\def\PY@tc##1{\textcolor[rgb]{0.00,0.00,1.00}{##1}}}
\@namedef{PY@tok@nc}{\let\PY@bf=\textbf\def\PY@tc##1{\textcolor[rgb]{0.00,0.00,1.00}{##1}}}
\@namedef{PY@tok@nn}{\let\PY@bf=\textbf\def\PY@tc##1{\textcolor[rgb]{0.00,0.00,1.00}{##1}}}
\@namedef{PY@tok@ne}{\let\PY@bf=\textbf\def\PY@tc##1{\textcolor[rgb]{0.80,0.25,0.22}{##1}}}
\@namedef{PY@tok@nv}{\def\PY@tc##1{\textcolor[rgb]{0.10,0.09,0.49}{##1}}}
\@namedef{PY@tok@no}{\def\PY@tc##1{\textcolor[rgb]{0.53,0.00,0.00}{##1}}}
\@namedef{PY@tok@nl}{\def\PY@tc##1{\textcolor[rgb]{0.46,0.46,0.00}{##1}}}
\@namedef{PY@tok@ni}{\let\PY@bf=\textbf\def\PY@tc##1{\textcolor[rgb]{0.44,0.44,0.44}{##1}}}
\@namedef{PY@tok@na}{\def\PY@tc##1{\textcolor[rgb]{0.41,0.47,0.13}{##1}}}
\@namedef{PY@tok@nt}{\let\PY@bf=\textbf\def\PY@tc##1{\textcolor[rgb]{0.00,0.50,0.00}{##1}}}
\@namedef{PY@tok@nd}{\def\PY@tc##1{\textcolor[rgb]{0.67,0.13,1.00}{##1}}}
\@namedef{PY@tok@s}{\def\PY@tc##1{\textcolor[rgb]{0.73,0.13,0.13}{##1}}}
\@namedef{PY@tok@sd}{\let\PY@it=\textit\def\PY@tc##1{\textcolor[rgb]{0.73,0.13,0.13}{##1}}}
\@namedef{PY@tok@si}{\let\PY@bf=\textbf\def\PY@tc##1{\textcolor[rgb]{0.64,0.35,0.47}{##1}}}
\@namedef{PY@tok@se}{\let\PY@bf=\textbf\def\PY@tc##1{\textcolor[rgb]{0.67,0.36,0.12}{##1}}}
\@namedef{PY@tok@sr}{\def\PY@tc##1{\textcolor[rgb]{0.64,0.35,0.47}{##1}}}
\@namedef{PY@tok@ss}{\def\PY@tc##1{\textcolor[rgb]{0.10,0.09,0.49}{##1}}}
\@namedef{PY@tok@sx}{\def\PY@tc##1{\textcolor[rgb]{0.00,0.50,0.00}{##1}}}
\@namedef{PY@tok@m}{\def\PY@tc##1{\textcolor[rgb]{0.40,0.40,0.40}{##1}}}
\@namedef{PY@tok@gh}{\let\PY@bf=\textbf\def\PY@tc##1{\textcolor[rgb]{0.00,0.00,0.50}{##1}}}
\@namedef{PY@tok@gu}{\let\PY@bf=\textbf\def\PY@tc##1{\textcolor[rgb]{0.50,0.00,0.50}{##1}}}
\@namedef{PY@tok@gd}{\def\PY@tc##1{\textcolor[rgb]{0.63,0.00,0.00}{##1}}}
\@namedef{PY@tok@gi}{\def\PY@tc##1{\textcolor[rgb]{0.00,0.52,0.00}{##1}}}
\@namedef{PY@tok@gr}{\def\PY@tc##1{\textcolor[rgb]{0.89,0.00,0.00}{##1}}}
\@namedef{PY@tok@ge}{\let\PY@it=\textit}
\@namedef{PY@tok@gs}{\let\PY@bf=\textbf}
\@namedef{PY@tok@gp}{\let\PY@bf=\textbf\def\PY@tc##1{\textcolor[rgb]{0.00,0.00,0.50}{##1}}}
\@namedef{PY@tok@go}{\def\PY@tc##1{\textcolor[rgb]{0.44,0.44,0.44}{##1}}}
\@namedef{PY@tok@gt}{\def\PY@tc##1{\textcolor[rgb]{0.00,0.27,0.87}{##1}}}
\@namedef{PY@tok@err}{\def\PY@bc##1{{\setlength{\fboxsep}{\string -\fboxrule}\fcolorbox[rgb]{1.00,0.00,0.00}{1,1,1}{\strut ##1}}}}
\@namedef{PY@tok@kc}{\let\PY@bf=\textbf\def\PY@tc##1{\textcolor[rgb]{0.00,0.50,0.00}{##1}}}
\@namedef{PY@tok@kd}{\let\PY@bf=\textbf\def\PY@tc##1{\textcolor[rgb]{0.00,0.50,0.00}{##1}}}
\@namedef{PY@tok@kn}{\let\PY@bf=\textbf\def\PY@tc##1{\textcolor[rgb]{0.00,0.50,0.00}{##1}}}
\@namedef{PY@tok@kr}{\let\PY@bf=\textbf\def\PY@tc##1{\textcolor[rgb]{0.00,0.50,0.00}{##1}}}
\@namedef{PY@tok@bp}{\def\PY@tc##1{\textcolor[rgb]{0.00,0.50,0.00}{##1}}}
\@namedef{PY@tok@fm}{\def\PY@tc##1{\textcolor[rgb]{0.00,0.00,1.00}{##1}}}
\@namedef{PY@tok@vc}{\def\PY@tc##1{\textcolor[rgb]{0.10,0.09,0.49}{##1}}}
\@namedef{PY@tok@vg}{\def\PY@tc##1{\textcolor[rgb]{0.10,0.09,0.49}{##1}}}
\@namedef{PY@tok@vi}{\def\PY@tc##1{\textcolor[rgb]{0.10,0.09,0.49}{##1}}}
\@namedef{PY@tok@vm}{\def\PY@tc##1{\textcolor[rgb]{0.10,0.09,0.49}{##1}}}
\@namedef{PY@tok@sa}{\def\PY@tc##1{\textcolor[rgb]{0.73,0.13,0.13}{##1}}}
\@namedef{PY@tok@sb}{\def\PY@tc##1{\textcolor[rgb]{0.73,0.13,0.13}{##1}}}
\@namedef{PY@tok@sc}{\def\PY@tc##1{\textcolor[rgb]{0.73,0.13,0.13}{##1}}}
\@namedef{PY@tok@dl}{\def\PY@tc##1{\textcolor[rgb]{0.73,0.13,0.13}{##1}}}
\@namedef{PY@tok@s2}{\def\PY@tc##1{\textcolor[rgb]{0.73,0.13,0.13}{##1}}}
\@namedef{PY@tok@sh}{\def\PY@tc##1{\textcolor[rgb]{0.73,0.13,0.13}{##1}}}
\@namedef{PY@tok@s1}{\def\PY@tc##1{\textcolor[rgb]{0.73,0.13,0.13}{##1}}}
\@namedef{PY@tok@mb}{\def\PY@tc##1{\textcolor[rgb]{0.40,0.40,0.40}{##1}}}
\@namedef{PY@tok@mf}{\def\PY@tc##1{\textcolor[rgb]{0.40,0.40,0.40}{##1}}}
\@namedef{PY@tok@mh}{\def\PY@tc##1{\textcolor[rgb]{0.40,0.40,0.40}{##1}}}
\@namedef{PY@tok@mi}{\def\PY@tc##1{\textcolor[rgb]{0.40,0.40,0.40}{##1}}}
\@namedef{PY@tok@il}{\def\PY@tc##1{\textcolor[rgb]{0.40,0.40,0.40}{##1}}}
\@namedef{PY@tok@mo}{\def\PY@tc##1{\textcolor[rgb]{0.40,0.40,0.40}{##1}}}
\@namedef{PY@tok@ch}{\let\PY@it=\textit\def\PY@tc##1{\textcolor[rgb]{0.24,0.48,0.48}{##1}}}
\@namedef{PY@tok@cm}{\let\PY@it=\textit\def\PY@tc##1{\textcolor[rgb]{0.24,0.48,0.48}{##1}}}
\@namedef{PY@tok@cpf}{\let\PY@it=\textit\def\PY@tc##1{\textcolor[rgb]{0.24,0.48,0.48}{##1}}}
\@namedef{PY@tok@c1}{\let\PY@it=\textit\def\PY@tc##1{\textcolor[rgb]{0.24,0.48,0.48}{##1}}}
\@namedef{PY@tok@cs}{\let\PY@it=\textit\def\PY@tc##1{\textcolor[rgb]{0.24,0.48,0.48}{##1}}}

\def\PYZbs{\char`\\}
\def\PYZus{\char`\_}
\def\PYZob{\char`\{}
\def\PYZcb{\char`\}}
\def\PYZca{\char`\^}
\def\PYZam{\char`\&}
\def\PYZlt{\char`\<}
\def\PYZgt{\char`\>}
\def\PYZsh{\char`\#}
\def\PYZpc{\char`\%}
\def\PYZdl{\char`\$}
\def\PYZhy{\char`\-}
\def\PYZsq{\char`\'}
\def\PYZdq{\char`\"}
\def\PYZti{\char`\~}
% for compatibility with earlier versions
\def\PYZat{@}
\def\PYZlb{[}
\def\PYZrb{]}
\makeatother


% For linebreaks inside Verbatim environment from package fancyvrb.
\makeatletter
		\newbox\Wrappedcontinuationbox
		\newbox\Wrappedvisiblespacebox
		\newcommand*\Wrappedvisiblespace {\textcolor{red}{\textvisiblespace}}
		\newcommand*\Wrappedcontinuationsymbol {\textcolor{red}{\llap{\tiny$\m@th\hookrightarrow$}}}
		\newcommand*\Wrappedcontinuationindent {3ex }
		\newcommand*\Wrappedafterbreak {\kern\Wrappedcontinuationindent\copy\Wrappedcontinuationbox}
		% Take advantage of the already applied Pygments mark-up to insert
		% potential linebreaks for TeX processing.
		%        {, <, #, %, $, ' and ": go to next line.
		%        _, }, ^, &, >, - and ~: stay at end of broken line.
		% Use of \textquotesingle for straight quote.
		\newcommand*\Wrappedbreaksatspecials {%
				\def\PYGZus{\discretionary{\char`\_}{\Wrappedafterbreak}{\char`\_}}%
				\def\PYGZob{\discretionary{}{\Wrappedafterbreak\char`\{}{\char`\{}}%
				\def\PYGZcb{\discretionary{\char`\}}{\Wrappedafterbreak}{\char`\}}}%
				\def\PYGZca{\discretionary{\char`\^}{\Wrappedafterbreak}{\char`\^}}%
				\def\PYGZam{\discretionary{\char`\&}{\Wrappedafterbreak}{\char`\&}}%
				\def\PYGZlt{\discretionary{}{\Wrappedafterbreak\char`\<}{\char`\<}}%
				\def\PYGZgt{\discretionary{\char`\>}{\Wrappedafterbreak}{\char`\>}}%
				\def\PYGZsh{\discretionary{}{\Wrappedafterbreak\char`\#}{\char`\#}}%
				\def\PYGZpc{\discretionary{}{\Wrappedafterbreak\char`\%}{\char`\%}}%
				\def\PYGZdl{\discretionary{}{\Wrappedafterbreak\char`\$}{\char`\$}}%
				\def\PYGZhy{\discretionary{\char`\-}{\Wrappedafterbreak}{\char`\-}}%
				\def\PYGZsq{\discretionary{}{\Wrappedafterbreak\textquotesingle}{\textquotesingle}}%
				\def\PYGZdq{\discretionary{}{\Wrappedafterbreak\char`\"}{\char`\"}}%
				\def\PYGZti{\discretionary{\char`\~}{\Wrappedafterbreak}{\char`\~}}%
		}
		% Some characters . , ; ? ! / are not pygmentized.
		% This macro makes them "active" and they will insert potential linebreaks
		\newcommand*\Wrappedbreaksatpunct {%
				\lccode`\~`\.\lowercase{\def~}{\discretionary{\hbox{\char`\.}}{\Wrappedafterbreak}{\hbox{\char`\.}}}%
				\lccode`\~`\,\lowercase{\def~}{\discretionary{\hbox{\char`\,}}{\Wrappedafterbreak}{\hbox{\char`\,}}}%
				\lccode`\~`\;\lowercase{\def~}{\discretionary{\hbox{\char`\;}}{\Wrappedafterbreak}{\hbox{\char`\;}}}%
				\lccode`\~`\:\lowercase{\def~}{\discretionary{\hbox{\char`\:}}{\Wrappedafterbreak}{\hbox{\char`\:}}}%
				\lccode`\~`\?\lowercase{\def~}{\discretionary{\hbox{\char`\?}}{\Wrappedafterbreak}{\hbox{\char`\?}}}%
				\lccode`\~`\!\lowercase{\def~}{\discretionary{\hbox{\char`\!}}{\Wrappedafterbreak}{\hbox{\char`\!}}}%
				\lccode`\~`\/\lowercase{\def~}{\discretionary{\hbox{\char`\/}}{\Wrappedafterbreak}{\hbox{\char`\/}}}%
				\catcode`\.\active
				\catcode`\,\active
				\catcode`\;\active
				\catcode`\:\active
				\catcode`\?\active
				\catcode`\!\active
				\catcode`\/\active
				\lccode`\~`\~
		}
\makeatother

\let\OriginalVerbatim=\Verbatim
\makeatletter
\renewcommand{\Verbatim}[1][1]{%
		%\parskip\z@skip
		\sbox\Wrappedcontinuationbox {\Wrappedcontinuationsymbol}%
		\sbox\Wrappedvisiblespacebox {\FV@SetupFont\Wrappedvisiblespace}%
		\def\FancyVerbFormatLine ##1{\hsize\linewidth
				\vtop{\raggedright\hyphenpenalty\z@\exhyphenpenalty\z@
						\doublehyphendemerits\z@\finalhyphendemerits\z@
						\strut ##1\strut}%
		}%
		% If the linebreak is at a space, the latter will be displayed as visible
		% space at end of first line, and a continuation symbol starts next line.
		% Stretch/shrink are however usually zero for typewriter font.
		\def\FV@Space {%
				\nobreak\hskip\z@ plus\fontdimen3\font minus\fontdimen4\font
				\discretionary{\copy\Wrappedvisiblespacebox}{\Wrappedafterbreak}
				{\kern\fontdimen2\font}%
		}%

		% Allow breaks at special characters using \PYG... macros.
		\Wrappedbreaksatspecials
		% Breaks at punctuation characters . , ; ? ! and / need catcode=\active
		\OriginalVerbatim[#1,codes*=\Wrappedbreaksatpunct]%
}
\makeatother

% Exact colors from NB
\definecolor{incolor}{HTML}{303F9F}
\definecolor{outcolor}{HTML}{D84315}
\definecolor{cellborder}{HTML}{CFCFCF}
\definecolor{cellbackground}{HTML}{F7F7F7}

% prompt
\makeatletter
\newcommand{\boxspacing}{\kern\kvtcb@left@rule\kern\kvtcb@boxsep}
\makeatother
\newcommand{\prompt}[4]{
		{\ttfamily\llap{{\color{#2}[#3]:\hspace{3pt}#4}}\vspace{-\baselineskip}}
}



% Prevent overflowing lines due to hard-to-break entities
\sloppy
% Setup hyperref package
\hypersetup{
	breaklinks=true,  % so long urls are correctly broken across lines
	colorlinks=true,
	urlcolor=urlcolor,
	linkcolor=linkcolor,
	citecolor=citecolor,
	}
% Slightly bigger margins than the latex defaults

\geometry{verbose,tmargin=1in,bmargin=1in,lmargin=1in,rmargin=1in}




% ---
% Informações de dados para CAPA e FOLHA DE ROSTO
% ---
\titulo{Projeto Final - Fitbit Fitness Data}
\autor{Daniel Terra Gomes}
\local{Campos dos Goytacazes, RJ}
\data{\today, v1.0.0}
%\orientador{Manuel Antonio Molina Palma}
%\coorientador{Equipe \abnTeX}
\instituicao{%
Universidade Estadual do Norte Fluminense Darcy Ribeiro
  \par
  Ciência da Computação
  \par
  IA 2022.3}
\tipotrabalho{Projeto de Pesquisa}
% O preambulo deve conter o tipo do trabalho, o objetivo,
% o nome da instituição e a área de concentração
\preambulo{Relatório Projeto Final apresentado ao Curso de Ciência da Computação da
Universidade Estadual do Norte Fluminense
Darcy Ribeiro, como requisito avaliativo da
disciplina.}
% ---


% ---
% Configurações de aparência do PDF final

% alterando o aspecto da cor azul
\definecolor{blue}{RGB}{41,5,195}

% informações do PDF
\makeatletter
\hypersetup{
     	%pagebackref=true,
		pdftitle={\@title},
		pdfauthor={\@author},
    	pdfsubject={\imprimirpreambulo},
	    pdfcreator={LaTeX with abnTeX2},
		pdfkeywords={abnt}{latex}{abntex}{abntex2}{trabalho acadêmico},
		colorlinks=true,       		% false: boxed links; true: colored links
    	linkcolor=blue,          	% color of internal links
    	citecolor=blue,        		% color of links to bibliography
    	filecolor=magenta,      		% color of file links
		urlcolor=blue,
		bookmarksdepth=4
}
\makeatother
% ---

% ---
% Espaçamentos entre linhas e parágrafos
% ---

% O tamanho do parágrafo é dado por:
\setlength{\parindent}{1.3cm}

% Controle do espaçamento entre um parágrafo e outro:
\setlength{\parskip}{0.2cm}  % tente também \onelineskip

% ---
% compila o indice
% ---
\makeindex
% ---
%\setcounter{chapter}{1}% Not using chapters, but they're used in the counters


% ----
% Início do documento
% ----
\begin{document}

% Seleciona o idioma do documento (conforme pacotes do babel)
%\selectlanguage{english}
\selectlanguage{brazil}

% Retira espaço extra obsoleto entre as frases.
\frenchspacing

% ----------------------------------------------------------
% ELEMENTOS PRÉ-TEXTUAIS
% ----------------------------------------------------------
% \pretextual

% ---
% Capa
% ---
\imprimircapa
% ---

% ---
% Folha de rosto
% (o * indica que haverá a ficha bibliográfica)
% ---
% ---

% ---
% Inserir a ficha bibliografica
% ---

% Isto é um exemplo de Ficha Catalográfica, ou ``Dados internacionais de
% catalogação-na-publicação''. Você pode utilizar este modelo como referência.
% Porém, provavelmente a biblioteca da sua universidade lhe fornecerá um PDF
% com a ficha catalográfica definitiva após a defesa do trabalho. Quando estiver
% com o documento, salve-o como PDF no diretório do seu projeto e substitua todo
% o conteúdo de implementação deste arquivo pelo comando abaixo:
%
% \begin{fichacatalografica}
%     \includepdf{fig_ficha_catalografica.pdf}
% \end{fichacatalografica}

%\begin{fichacatalografica}
%	\sffamily
%	\vspace*{\fill}					% Posição vertical
%	\begin{center}					% Minipage Centralizado
%	\fbox{\begin{minipage}[c][8cm]{13.5cm}		% Largura
%	\small
%	\imprimirautor
%	%Sobrenome, Nome do autor
%
%	\hspace{0.5cm} \imprimirtitulo  / \imprimirautor. --
%	\imprimirlocal, \imprimirdata-
%
%	\hspace{0.5cm} \pageref{LastPage} p. : il. \\ % (algumas color.) ; 30 cm.\\
%
%	\hspace{0.5cm} \imprimirorientadorRotulo~\imprimirorientador\\
%
%	\hspace{0.5cm}
%	\parbox[t]{\textwidth}{\imprimirtipotrabalho~--~\imprimirinstituicao,
%	\imprimirdata.}\\
%
%	\hspace{0.5cm}
%		1. Veículos autônomos.
%		2. Inteligência Artificial.
%		3. Machine Learning.
%		4. Condução Autônoma.
%		I. Manuel Antonio Molina Palma.
%		II. Universidade Estadual do Norte Fluminense Darcy Ribeiro.
%		III. Faculdade de Ciência da Computação.
%		IV. Veículos autônomos e inteligência artificial:
% um estudo sobre a implementação no brasil.
%	\end{minipage}}
%	\end{center}
%\end{fichacatalografica}
% ---

% ---
% Inserir errata
% ---

% Inserir folha de aprovação
% ---

% Isto é um exemplo de Folha de aprovação, elemento obrigatório da NBR
% 14724/2011 (seção 4.2.1.3). Você pode utilizar este modelo até a aprovação
% do trabalho. Após isso, substitua todo o conteúdo deste arquivo por uma
% imagem da página assinada pela banca com o comando abaixo:
%
% \includepdf{folhadeaprovacao_final.pdf}
%
\begin{folhadeaprovacao}

	\begin{center}
		{\ABNTEXchapterfont\large\imprimirautor}

		\vspace*{\fill}\vspace*{\fill}
		\begin{center}
			\ABNTEXchapterfont\bfseries\Large\imprimirtitulo
		\end{center}
		\vspace*{\fill}

		\hspace{.45\textwidth}
		\begin{minipage}{.5\textwidth}
			\imprimirpreambulo
		\end{minipage}%
		\vspace*{\fill}
	\end{center}
	% \begin{center}
	%	\imprimirlocal, \today
	%\end{center}
	%%%%%%%%%%%%\assinatura{\textbf{\imprimirorientador} \\ Orientador}

	%\assinatura{\textbf{Professor} \\ Convidado 1}
	% \assinatura{\textbf{Professor} \\ Convidado 2}
	%\assinatura{\textbf{Professor} \\ Convidado 3}
	%\assinatura{\textbf{Professor} \\ Convidado 4}

	\begin{center}
		\vspace*{0.5cm}
		{\large\imprimirlocal}
		\par
		{\large\imprimirdata}
		\vspace*{1cm}
	\end{center}

\end{folhadeaprovacao}
% ---

% ---
% Dedicatória
% ---

% ---
% Agradecimentos
% ---
%%%%\begin{agradecimentos}
%%%%Agradeço aos meus pais que se dedicaram para que eu pudesse estar cursando esta graduação, assim podendo completar mais uma etapa da minha vida.
%%%%Sem o apoio, conselhos, carinho e amor, nada disso seria possível. Sou eternamente grato por tudo que vocês fazem e sempre fizeram para que minha vida fosse especial.
%%%%
%%%%Agradeço ao professor Dr. Manuel Antonio Molina Palma pela dedicação e paciência durante o lecionamento desta disciplina, e obrigado pela ajuda e por estar disponível nos momentos de necessidade.
%%%%
%%%%Por último, mas não menos importante, agradeço a toda minha família e amigos que estiveram comigo em todos os momentos da minha vida.
%%%%
%%%%
%%%%\end{agradecimentos}
% ---

% ---
% Epígrafe
% -
% ---
% RESUMOS
% ---

% resumo em português

%%%%%%%%%%%
% ---
% inserir lista de ilustrações
% ---

% ---

% ---


% ---
% inserir lista de símbolos
% ---


% ---
% inserir o sumario
% ---
\pdfbookmark[0]{\contentsname}{toc}
\tableofcontents*
\cleardoublepage
% ---


% ----------------------------------------------------------
% ELEMENTOS TEXTUAIS
% ----------------------------------------------------------
\textual

% ----------------------------------------------------------
% Introdução (exemplo de capítulo sem numeração, mas presente no Sumário)
% ----------------------------------------------------------


% PARTE
% ----------------------------------------------------------
%\part{Preparação da pesquisa}
% ----------------------------------------------------------

% ---
% Capitulo com exemplos de comandos inseridos de arquivo externo
% ---
%\include{abntex2-modelo-include-comandos}
\chapter{Projeto Final - Fitbit Fitness Data}

\section{Definição do problema.}

Quais são algumas tendências no uso de dispositivos inteligentes?

Os dados que iremos usar para responder essa pergunta é o FitBit Fitness
Tracker Data(License CC0: Public Domain, disponível através do Mobius
distribuído através do Amazon Mechanical Turk entre 12 de março de 2016
e 12 de maio de 2016.)

O conjunto de dados contém rastreamento de atividade física pessoal para
33 usuários do Fitbit. Esses usuários qualificados do Fitbit concordaram
com o envio de dados de rastreamento pessoal, incluindo minutos de
desempenho de condicionamento físico, frequência cardíaca e
monitoramento do sono.

O aplicativo fornece aos usuários dados de saúde relacionados à
atividade, sono, estresse, ciclo menstrual e hábitos de foco. Esses
dados podem ajudar os usuários a entender melhor seus hábitos atuais e
tomar decisões saudáveis. O aplicativo se conecta à sua linha de
produtos de bem-estar inteligentes da empresa.

Temos como \textbf{objetivo} nesse \textbf{projeto} analisar os dados de
uso de dispositivos inteligentes para obter informações sobre como as
pessoas já estão usando seus dispositivos e usar essas tendências para
entendermos e identificar possiveis correlações e assim propor possiveis
hypothesis para o frame de dados analisado.


\subsection{Bibliotecas que iremos usar no nosso Projeto}

\begin{tcolorbox}[breakable, size=fbox, boxrule=1pt, pad at break*=1mm,colback=cellbackground, colframe=cellborder]
    \prompt{In}{incolor}{1}{\boxspacing}
    \begin{Verbatim}[commandchars=\\\{\}]
        \PY{n+nf}{library}\PY{p}{(}\PY{n}{janitor}\PY{p}{)}\PY{+w}{ }\PY{c+c1}{\PYZsh{} janitor tem pequenas ferramentas simples para examinar e limpar dados sujos.}
        \PY{n+nf}{library}\PY{p}{(}\PY{n}{arrow}\PY{p}{)}\PY{+w}{ }\PY{c+c1}{\PYZsh{} permite que os usuários leiam e gravem dados em vários formatos: Parquet, csv, JSON}
        \PY{n+nf}{library}\PY{p}{(}\PY{n}{tidyverse}\PY{p}{)}\PY{+w}{ }\PY{c+c1}{\PYZsh{} você disse ciência de dados usando R?}
        \PY{n+nf}{library}\PY{p}{(}\PY{n}{naniar}\PY{p}{)}\PY{+w}{ }\PY{c+c1}{\PYZsh{} lidando com valores NA}
        \PY{n+nf}{library}\PY{p}{(}\PY{n}{ggsci}\PY{p}{)}\PY{+w}{ }\PY{c+c1}{\PYZsh{} Revista Científica e Tema Sci\PYZhy{}Fi Paletas de cores para ggplot2}
        \PY{n+nf}{library}\PY{p}{(}\PY{n}{skimr}\PY{p}{)}\PY{+w}{ }\PY{c+c1}{\PYZsh{} Skim estatísticas de resumo úteis}
        \PY{n+nf}{library}\PY{p}{(}\PY{n}{lubridate}\PY{p}{)}\PY{+w}{ }\PY{c+c1}{\PYZsh{} Lubridate fornece ferramentas que facilitam a análise e manipulação de datas. ymd\PYZus{}hms() week()}
        \PY{n+nf}{library}\PY{p}{(}\PY{n}{ggpubr}\PY{p}{)}\PY{+w}{ }\PY{c+c1}{\PYZsh{} O pacote \PYZsq{}ggpubr\PYZsq{} fornece algumas funções fáceis de usar para criar e personalizar gráficos prontos para publicação baseados em \PYZsq{}ggplot2\PYZsq{}.stat\PYZus{}cor}
    \end{Verbatim}
\end{tcolorbox}

\begin{Verbatim}[commandchars=\\\{\}]

    Attaching package: ‘janitor’


    The following objects are masked from ‘package:stats’:

    chisq.test, fisher.test



    Attaching package: ‘arrow’


    The following object is masked from ‘package:utils’:

    timestamp


    ── \textbf{Attaching packages} ─────────────────────────────────────── tidyverse
    1.3.2 ──
    \textcolor{ansi-green}{✔} \textcolor{ansi-blue}{ggplot2} 3.4.0      \textcolor{ansi-green}{✔} \textcolor{ansi-blue}{purrr  } 1.0.1
    \textcolor{ansi-green}{✔} \textcolor{ansi-blue}{tibble } 3.1.8      \textcolor{ansi-green}{✔} \textcolor{ansi-blue}{dplyr  } 1.0.10
    \textcolor{ansi-green}{✔} \textcolor{ansi-blue}{tidyr  } 1.2.1      \textcolor{ansi-green}{✔} \textcolor{ansi-blue}{stringr} 1.5.0
    \textcolor{ansi-green}{✔} \textcolor{ansi-blue}{readr  } 2.1.3      \textcolor{ansi-green}{✔} \textcolor{ansi-blue}{forcats} 0.5.2
    ── \textbf{Conflicts} ──────────────────────────────────────────
    tidyverse\_conflicts() ──
    \textcolor{ansi-red}{✖} \textcolor{ansi-blue}{dplyr}::\textcolor{ansi-green}{filter()} masks \textcolor{ansi-blue}{stats}::filter()
    \textcolor{ansi-red}{✖} \textcolor{ansi-blue}{dplyr}::\textcolor{ansi-green}{lag()}    masks \textcolor{ansi-blue}{stats}::lag()

    Attaching package: ‘skimr’


    The following object is masked from ‘package:naniar’:

    n\_complete


    Loading required package: timechange


    Attaching package: ‘lubridate’


    The following object is masked from ‘package:arrow’:

    duration


    The following objects are masked from ‘package:base’:

    date, intersect, setdiff, union


\end{Verbatim}

\section{Apresentação dos dados}

Como o objetivo é identificar tendências no uso de dispositivos
inteligentes, decidimos trabalhar com os cinco dataframes a seguir:

\begin{itemize}
    \tightlist
    \item
          sleepDay\_merged.csv
    \item
          dailyActivity\_merged.csv
    \item
          dailyIntensities\_merged.csv
    \item
          hourlyIntensities\_merged.csv
    \item
          hourlyCalories\_merged.csv
\end{itemize}

\subsection{Lendo os arquivos .csv}

\begin{tcolorbox}[breakable, size=fbox, boxrule=1pt, pad at break*=1mm,colback=cellbackground, colframe=cellborder]
    \prompt{In}{incolor}{2}{\boxspacing}
    \begin{Verbatim}[commandchars=\\\{\}]
        \PY{n}{sleep\PYZus{}day\PYZus{}file}\PY{+w}{ }\PY{o}{\PYZlt{}\PYZhy{}}\PY{+w}{ }\PY{n+nf}{read\PYZus{}csv}\PY{p}{(}\PY{l+s}{\PYZdq{}}\PY{l+s}{/kaggle/input/fitbit/Fitabase Data 4.12.16\PYZhy{}5.12.16/sleepDay\PYZus{}merged.csv\PYZdq{}}\PY{p}{)}
        \PY{n}{daily\PYZus{}activity\PYZus{}file}\PY{+w}{ }\PY{o}{\PYZlt{}\PYZhy{}}\PY{+w}{ }\PY{n+nf}{read\PYZus{}csv}\PY{p}{(}\PY{l+s}{\PYZdq{}}\PY{l+s}{/kaggle/input/fitbit/Fitabase Data 4.12.16\PYZhy{}5.12.16/dailyActivity\PYZus{}merged.csv\PYZdq{}}\PY{p}{)}
        \PY{n}{daily\PYZus{}intensities\PYZus{}file}\PY{+w}{ }\PY{o}{\PYZlt{}\PYZhy{}}\PY{+w}{ }\PY{n+nf}{read\PYZus{}csv}\PY{p}{(}\PY{l+s}{\PYZdq{}}\PY{l+s}{/kaggle/input/fitbit/Fitabase Data 4.12.16\PYZhy{}5.12.16/dailyIntensities\PYZus{}merged.csv\PYZdq{}}\PY{p}{)}
        \PY{n}{hourly\PYZus{}intensities\PYZus{}file}\PY{+w}{ }\PY{o}{\PYZlt{}\PYZhy{}}\PY{+w}{ }\PY{n+nf}{read\PYZus{}csv}\PY{p}{(}\PY{l+s}{\PYZdq{}}\PY{l+s}{/kaggle/input/fitbit/Fitabase Data 4.12.16\PYZhy{}5.12.16/hourlyIntensities\PYZus{}merged.csv\PYZdq{}}\PY{p}{)}
        \PY{n}{hourly\PYZus{}calories\PYZus{}file}\PY{+w}{ }\PY{o}{\PYZlt{}\PYZhy{}}\PY{+w}{ }\PY{n+nf}{read\PYZus{}csv}\PY{p}{(}\PY{l+s}{\PYZdq{}}\PY{l+s}{/kaggle/input/fitbit/Fitabase Data 4.12.16\PYZhy{}5.12.16/hourlyCalories\PYZus{}merged.csv\PYZdq{}}\PY{p}{)}
    \end{Verbatim}
\end{tcolorbox}

\begin{Verbatim}[commandchars=\\\{\}]
    \textbf{Rows: }\textcolor{ansi-blue}{413} \textbf{Columns: }\textcolor{ansi-blue}{5}
    \textcolor{ansi-cyan}{──} \textbf{Column specification}
    \textcolor{ansi-cyan}{────────────────────────────────────────────────────────}
    \textbf{Delimiter:} ","
    \textcolor{ansi-red}{chr} (1): SleepDay
    \textcolor{ansi-green}{dbl} (4): Id, TotalSleepRecords, TotalMinutesAsleep, TotalTimeInBed

    \textcolor{ansi-cyan}{ℹ} Use `spec()` to retrieve the full column specification for this
    data.
    \textcolor{ansi-cyan}{ℹ} Specify the column types or set `show\_col\_types = FALSE` to quiet
    this message.
    \textbf{Rows: }\textcolor{ansi-blue}{940} \textbf{Columns: }\textcolor{ansi-blue}{15}
    \textcolor{ansi-cyan}{──} \textbf{Column specification}
    \textcolor{ansi-cyan}{────────────────────────────────────────────────────────}
    \textbf{Delimiter:} ","
    \textcolor{ansi-red}{chr}  (1): ActivityDate
    \textcolor{ansi-green}{dbl} (14): Id, TotalSteps, TotalDistance, TrackerDistance,
    LoggedActivitiesDi{\ldots}

    \textcolor{ansi-cyan}{ℹ} Use `spec()` to retrieve the full column specification for this
    data.
    \textcolor{ansi-cyan}{ℹ} Specify the column types or set `show\_col\_types = FALSE` to quiet
    this message.
    \textbf{Rows: }\textcolor{ansi-blue}{940} \textbf{Columns: }\textcolor{ansi-blue}{10}
    \textcolor{ansi-cyan}{──} \textbf{Column specification}
    \textcolor{ansi-cyan}{────────────────────────────────────────────────────────}
    \textbf{Delimiter:} ","
    \textcolor{ansi-red}{chr} (1): ActivityDay
    \textcolor{ansi-green}{dbl} (9): Id, SedentaryMinutes, LightlyActiveMinutes,
    FairlyActiveMinutes, Ve{\ldots}

    \textcolor{ansi-cyan}{ℹ} Use `spec()` to retrieve the full column specification for this
    data.
    \textcolor{ansi-cyan}{ℹ} Specify the column types or set `show\_col\_types = FALSE` to quiet
    this message.
    \textbf{Rows: }\textcolor{ansi-blue}{22099} \textbf{Columns: }\textcolor{ansi-blue}{4}
    \textcolor{ansi-cyan}{──} \textbf{Column specification}
    \textcolor{ansi-cyan}{────────────────────────────────────────────────────────}
    \textbf{Delimiter:} ","
    \textcolor{ansi-red}{chr} (1): ActivityHour
    \textcolor{ansi-green}{dbl} (3): Id, TotalIntensity, AverageIntensity

    \textcolor{ansi-cyan}{ℹ} Use `spec()` to retrieve the full column specification for this
    data.
    \textcolor{ansi-cyan}{ℹ} Specify the column types or set `show\_col\_types = FALSE` to quiet
    this message.
    \textbf{Rows: }\textcolor{ansi-blue}{22099} \textbf{Columns: }\textcolor{ansi-blue}{3}
    \textcolor{ansi-cyan}{──} \textbf{Column specification}
    \textcolor{ansi-cyan}{────────────────────────────────────────────────────────}
    \textbf{Delimiter:} ","
    \textcolor{ansi-red}{chr} (1): ActivityHour
    \textcolor{ansi-green}{dbl} (2): Id, Calories

    \textcolor{ansi-cyan}{ℹ} Use `spec()` to retrieve the full column specification for this
    data.
    \textcolor{ansi-cyan}{ℹ} Specify the column types or set `show\_col\_types = FALSE` to quiet
    this message.
\end{Verbatim}

\subsection{Convertendo os arquivos .csv em .parquet}

A partir dessa conversão iremos ganhar desempenho em nossas análises

O \href{https://www.upsolver.com/blog/apache-parquet-why-use}{Apache
    Parquet} é um formato de arquivo projetado para oferecer suporte ao
processamento rápido de dados complexos, com várias características
notáveis: * Compressão * Evolução do esquema * Código aberto e não
proprietário Mesmo que o uso do Parquet não seja necessário para o nosso
conjunto de dados, pois não é um banco muito grande. Ainda sim nos trara
um pouco mais de rapidez.

\begin{center}
    \adjustimage{max size={0.9\linewidth}{0.9\paperheight}}{finalproject_files/pa.png}
\end{center}

\href{https://blog.openbridge.com/how-to-be-a-hero-with-powerful-parquet-google-and-amazon-f2ae0f35ee04}{ref.}

\begin{tcolorbox}[breakable, size=fbox, boxrule=1pt, pad at break*=1mm,colback=cellbackground, colframe=cellborder]
    \prompt{In}{incolor}{3}{\boxspacing}
    \begin{Verbatim}[commandchars=\\\{\}]
        \PY{n+nf}{write\PYZus{}parquet}\PY{p}{(}\PY{n}{sleep\PYZus{}day\PYZus{}file}\PY{p}{,}\PY{+w}{ }\PY{l+s}{\PYZdq{}}\PY{l+s}{/kaggle/working/sleepDay.parquet\PYZdq{}}\PY{p}{)}
        \PY{n+nf}{write\PYZus{}parquet}\PY{p}{(}\PY{n}{daily\PYZus{}activity\PYZus{}file}\PY{p}{,}\PY{+w}{ }\PY{l+s}{\PYZdq{}}\PY{l+s}{/kaggle/working/dailyActivity.parquet\PYZdq{}}\PY{p}{)}
        \PY{n+nf}{write\PYZus{}parquet}\PY{p}{(}\PY{n}{daily\PYZus{}intensities\PYZus{}file}\PY{p}{,}\PY{+w}{ }\PY{l+s}{\PYZdq{}}\PY{l+s}{/kaggle/working/dailyIntensities.parquet\PYZdq{}}\PY{p}{)}
        \PY{n+nf}{write\PYZus{}parquet}\PY{p}{(}\PY{n}{hourly\PYZus{}intensities\PYZus{}file}\PY{p}{,}\PY{+w}{ }\PY{l+s}{\PYZdq{}}\PY{l+s}{/kaggle/working/hourly\PYZus{}intensities.parquet\PYZdq{}}\PY{p}{)}
        \PY{n+nf}{write\PYZus{}parquet}\PY{p}{(}\PY{n}{hourly\PYZus{}calories\PYZus{}file}\PY{p}{,}\PY{+w}{ }\PY{l+s}{\PYZdq{}}\PY{l+s}{/kaggle/working/hourly\PYZus{}calories.parquet\PYZdq{}}\PY{p}{)}
    \end{Verbatim}
\end{tcolorbox}

\subsection{Lendo os arquivos .parquet}

\begin{tcolorbox}[breakable, size=fbox, boxrule=1pt, pad at break*=1mm,colback=cellbackground, colframe=cellborder]
    \prompt{In}{incolor}{4}{\boxspacing}
    \begin{Verbatim}[commandchars=\\\{\}]
        \PY{n}{sleep\PYZus{}day\PYZus{}file}\PY{+w}{ }\PY{o}{\PYZlt{}\PYZhy{}}\PY{+w}{ }\PY{n+nf}{read\PYZus{}parquet}\PY{p}{(}\PY{l+s}{\PYZdq{}}\PY{l+s}{/kaggle/working/sleepDay.parquet\PYZdq{}}\PY{p}{)}
        \PY{n}{daily\PYZus{}activity\PYZus{}file}\PY{+w}{ }\PY{o}{\PYZlt{}\PYZhy{}}\PY{+w}{ }\PY{n+nf}{read\PYZus{}parquet}\PY{p}{(}\PY{l+s}{\PYZdq{}}\PY{l+s}{/kaggle/working/dailyActivity.parquet\PYZdq{}}\PY{p}{)}
        \PY{n}{daily\PYZus{}intensities\PYZus{}file}\PY{+w}{ }\PY{o}{\PYZlt{}\PYZhy{}}\PY{+w}{ }\PY{n+nf}{read\PYZus{}parquet}\PY{p}{(}\PY{l+s}{\PYZdq{}}\PY{l+s}{/kaggle/working/dailyIntensities.parquet\PYZdq{}}\PY{p}{)}
        \PY{n}{hourly\PYZus{}intensities\PYZus{}file}\PY{+w}{ }\PY{o}{\PYZlt{}\PYZhy{}}\PY{+w}{ }\PY{n+nf}{read\PYZus{}parquet}\PY{p}{(}\PY{l+s}{\PYZdq{}}\PY{l+s}{/kaggle/working/hourly\PYZus{}intensities.parquet\PYZdq{}}\PY{p}{)}
        \PY{n}{hourly\PYZus{}calories\PYZus{}file}\PY{+w}{ }\PY{o}{\PYZlt{}\PYZhy{}}\PY{+w}{ }\PY{n+nf}{read\PYZus{}parquet}\PY{p}{(}\PY{l+s}{\PYZdq{}}\PY{l+s}{/kaggle/working/hourly\PYZus{}calories.parquet\PYZdq{}}\PY{p}{)}
    \end{Verbatim}
\end{tcolorbox}


\subsection{Entendendo um pouco dos dados}

Aqui teremos o nome primeiro contato com o formato, conjunto dos dados
que iremos trabalhar no nosso projeto

\begin{tcolorbox}[breakable, size=fbox, boxrule=1pt, pad at break*=1mm,colback=cellbackground, colframe=cellborder]
    \prompt{In}{incolor}{5}{\boxspacing}
    \begin{Verbatim}[commandchars=\\\{\}]
        \PY{c+c1}{\PYZsh{}skim\PYZus{}without\PYZus{}charts(sleep\PYZus{}day\PYZus{}file)}
        \PY{c+c1}{\PYZsh{}skim\PYZus{}without\PYZus{}charts(daily\PYZus{}activity\PYZus{}file)}
        \PY{c+c1}{\PYZsh{}skim\PYZus{}without\PYZus{}charts(daily\PYZus{}intensities\PYZus{}file)}
        \PY{c+c1}{\PYZsh{}skim\PYZus{}without\PYZus{}charts(hourly\PYZus{}intensities\PYZus{}file)}
        \PY{c+c1}{\PYZsh{}skim\PYZus{}without\PYZus{}charts(hourly\PYZus{}calories\PYZus{}file)}
    \end{Verbatim}
\end{tcolorbox}

skim output foi comentado pois não estava permitindo o push no kaggle.
Provavelmente, devido ao tamanho da saída.

\begin{tcolorbox}[breakable, size=fbox, boxrule=1pt, pad at break*=1mm,colback=cellbackground, colframe=cellborder]
    \prompt{In}{incolor}{6}{\boxspacing}
    \begin{Verbatim}[commandchars=\\\{\}]
        \PY{n+nf}{glimpse}\PY{p}{(}\PY{n}{sleep\PYZus{}day\PYZus{}file}\PY{p}{)}
        \PY{n+nf}{glimpse}\PY{p}{(}\PY{n}{daily\PYZus{}activity\PYZus{}file}\PY{p}{)}
        \PY{n+nf}{glimpse}\PY{p}{(}\PY{n}{daily\PYZus{}intensities\PYZus{}file}\PY{p}{)}
        \PY{n+nf}{glimpse}\PY{p}{(}\PY{n}{hourly\PYZus{}intensities\PYZus{}file}\PY{p}{)}
        \PY{n+nf}{glimpse}\PY{p}{(}\PY{n}{hourly\PYZus{}calories\PYZus{}file}\PY{p}{)}
    \end{Verbatim}
\end{tcolorbox}

\begin{Verbatim}[commandchars=\\\{\}]
    Rows: 413
    Columns: 5
    \$ Id                 \textcolor{ansi-black-intense}{<dbl>} 1503960366, 1503960366,
    1503960366, 1503960366, 150…
    \$ SleepDay           \textcolor{ansi-black-intense}{<chr>} "4/12/2016 12:00:00 AM",
    "4/13/2016 12:00:00 AM", "…
    \$ TotalSleepRecords  \textcolor{ansi-black-intense}{<dbl>} 1, 2, 1, 2, 1, 1, 1, 1, 1, 1, 1,
    1, 1, 1, 1, 1, 1, …
    \$ TotalMinutesAsleep \textcolor{ansi-black-intense}{<dbl>} 327, 384, 412, 340, 700, 304, 360,
    325, 361, 430, 2…
    \$ TotalTimeInBed     \textcolor{ansi-black-intense}{<dbl>} 346, 407, 442, 367, 712, 320, 377,
    364, 384, 449, 3…
    Rows: 940
    Columns: 15
    \$ Id                       \textcolor{ansi-black-intense}{<dbl>} 1503960366, 1503960366,
    1503960366, 150396036…
    \$ ActivityDate             \textcolor{ansi-black-intense}{<chr>} "4/12/2016", "4/13/2016",
    "4/14/2016", "4/15/…
    \$ TotalSteps               \textcolor{ansi-black-intense}{<dbl>} 13162, 10735, 10460, 9762,
    12669, 9705, 13019…
    \$ TotalDistance            \textcolor{ansi-black-intense}{<dbl>} 8.50, 6.97, 6.74, 6.28,
    8.16, 6.48, 8.59, 9.8…
    \$ TrackerDistance          \textcolor{ansi-black-intense}{<dbl>} 8.50, 6.97, 6.74, 6.28,
    8.16, 6.48, 8.59, 9.8…
    \$ LoggedActivitiesDistance \textcolor{ansi-black-intense}{<dbl>} 0, 0, 0, 0, 0, 0, 0, 0, 0,
    0, 0, 0, 0, 0, 0, …
    \$ VeryActiveDistance       \textcolor{ansi-black-intense}{<dbl>} 1.88, 1.57, 2.44, 2.14,
    2.71, 3.19, 3.25, 3.5…
    \$ ModeratelyActiveDistance \textcolor{ansi-black-intense}{<dbl>} 0.55, 0.69, 0.40, 1.26,
    0.41, 0.78, 0.64, 1.3…
    \$ LightActiveDistance      \textcolor{ansi-black-intense}{<dbl>} 6.06, 4.71, 3.91, 2.83,
    5.04, 2.51, 4.71, 5.0…
    \$ SedentaryActiveDistance  \textcolor{ansi-black-intense}{<dbl>} 0, 0, 0, 0, 0, 0, 0, 0, 0,
    0, 0, 0, 0, 0, 0, …
    \$ VeryActiveMinutes        \textcolor{ansi-black-intense}{<dbl>} 25, 21, 30, 29, 36, 38, 42,
    50, 28, 19, 66, 4…
    \$ FairlyActiveMinutes      \textcolor{ansi-black-intense}{<dbl>} 13, 19, 11, 34, 10, 20, 16,
    31, 12, 8, 27, 21…
    \$ LightlyActiveMinutes     \textcolor{ansi-black-intense}{<dbl>} 328, 217, 181, 209, 221,
    164, 233, 264, 205, …
    \$ SedentaryMinutes         \textcolor{ansi-black-intense}{<dbl>} 728, 776, 1218, 726, 773,
    539, 1149, 775, 818…
    \$ Calories                 \textcolor{ansi-black-intense}{<dbl>} 1985, 1797, 1776, 1745,
    1863, 1728, 1921, 203…
    Rows: 940
    Columns: 10
    \$ Id                       \textcolor{ansi-black-intense}{<dbl>} 1503960366, 1503960366,
    1503960366, 150396036…
    \$ ActivityDay              \textcolor{ansi-black-intense}{<chr>} "4/12/2016", "4/13/2016",
    "4/14/2016", "4/15/…
    \$ SedentaryMinutes         \textcolor{ansi-black-intense}{<dbl>} 728, 776, 1218, 726, 773,
    539, 1149, 775, 818…
    \$ LightlyActiveMinutes     \textcolor{ansi-black-intense}{<dbl>} 328, 217, 181, 209, 221,
    164, 233, 264, 205, …
    \$ FairlyActiveMinutes      \textcolor{ansi-black-intense}{<dbl>} 13, 19, 11, 34, 10, 20, 16,
    31, 12, 8, 27, 21…
    \$ VeryActiveMinutes        \textcolor{ansi-black-intense}{<dbl>} 25, 21, 30, 29, 36, 38, 42,
    50, 28, 19, 66, 4…
    \$ SedentaryActiveDistance  \textcolor{ansi-black-intense}{<dbl>} 0, 0, 0, 0, 0, 0, 0, 0, 0,
    0, 0, 0, 0, 0, 0, …
    \$ LightActiveDistance      \textcolor{ansi-black-intense}{<dbl>} 6.06, 4.71, 3.91, 2.83,
    5.04, 2.51, 4.71, 5.0…
    \$ ModeratelyActiveDistance \textcolor{ansi-black-intense}{<dbl>} 0.55, 0.69, 0.40, 1.26,
    0.41, 0.78, 0.64, 1.3…
    \$ VeryActiveDistance       \textcolor{ansi-black-intense}{<dbl>} 1.88, 1.57, 2.44, 2.14,
    2.71, 3.19, 3.25, 3.5…
    Rows: 22,099
    Columns: 4
    \$ Id               \textcolor{ansi-black-intense}{<dbl>} 1503960366, 1503960366, 1503960366,
    1503960366, 15039…
    \$ ActivityHour     \textcolor{ansi-black-intense}{<chr>} "4/12/2016 12:00:00 AM", "4/12/2016
    1:00:00 AM", "4/1…
    \$ TotalIntensity   \textcolor{ansi-black-intense}{<dbl>} 20, 8, 7, 0, 0, 0, 0, 0, 13, 30, 29,
    12, 11, 6, 36, 5…
    \$ AverageIntensity \textcolor{ansi-black-intense}{<dbl>} 0.333333, 0.133333, 0.116667,
    0.000000, 0.000000, 0.0…
    Rows: 22,099
    Columns: 3
    \$ Id           \textcolor{ansi-black-intense}{<dbl>} 1503960366, 1503960366, 1503960366,
    1503960366, 150396036…
    \$ ActivityHour \textcolor{ansi-black-intense}{<chr>} "4/12/2016 12:00:00 AM", "4/12/2016
    1:00:00 AM", "4/12/20…
    \$ Calories     \textcolor{ansi-black-intense}{<dbl>} 81, 61, 59, 47, 48, 48, 48, 47, 68, 141,
    99, 76, 73, 66, …
\end{Verbatim}

A partir desses resultados podemos identificar que não temos dados NA,
dados faltantes em nosso conjuntos de dados. Assim como podemos ter uma
média desses valores, e ver como as colunas estão organizadas.

\subsubsection{Número de usuários únicos}

\begin{tcolorbox}[breakable, size=fbox, boxrule=1pt, pad at break*=1mm,colback=cellbackground, colframe=cellborder]
    \prompt{In}{incolor}{7}{\boxspacing}
    \begin{Verbatim}[commandchars=\\\{\}]
        \PY{n+nf}{count}\PY{p}{(}\PY{n+nf}{distinct}\PY{p}{(}\PY{n}{sleep\PYZus{}day\PYZus{}file}\PY{p}{,}\PY{+w}{ }\PY{n}{Id}\PY{p}{)}\PY{p}{)}
        \PY{n+nf}{count}\PY{p}{(}\PY{n+nf}{distinct}\PY{p}{(}\PY{n}{daily\PYZus{}activity\PYZus{}file}\PY{p}{,}\PY{+w}{ }\PY{n}{Id}\PY{p}{)}\PY{p}{)}
        \PY{n+nf}{count}\PY{p}{(}\PY{n+nf}{distinct}\PY{p}{(}\PY{n}{daily\PYZus{}intensities\PYZus{}file}\PY{p}{,}\PY{+w}{ }\PY{n}{Id}\PY{p}{)}\PY{p}{)}
        \PY{n+nf}{count}\PY{p}{(}\PY{n+nf}{distinct}\PY{p}{(}\PY{n}{hourly\PYZus{}intensities\PYZus{}file}\PY{p}{,}\PY{+w}{ }\PY{n}{Id}\PY{p}{)}\PY{p}{)}
        \PY{n+nf}{count}\PY{p}{(}\PY{n+nf}{distinct}\PY{p}{(}\PY{n}{hourly\PYZus{}calories\PYZus{}file}\PY{p}{,}\PY{+w}{ }\PY{n}{Id}\PY{p}{)}\PY{p}{)}
    \end{Verbatim}
\end{tcolorbox}

A tibble: 1 × 1
\begin{tabular}{l}
    n     \\
    <int> \\
    \hline
    24    \\
\end{tabular}



A tibble: 1 × 1
\begin{tabular}{l}
    n     \\
    <int> \\
    \hline
    33    \\
\end{tabular}



A tibble: 1 × 1
\begin{tabular}{l}
    n     \\
    <int> \\
    \hline
    33    \\
\end{tabular}



A tibble: 1 × 1
\begin{tabular}{l}
    n     \\
    <int> \\
    \hline
    33    \\
\end{tabular}



A tibble: 1 × 1
\begin{tabular}{l}
    n     \\
    <int> \\
    \hline
    33    \\
\end{tabular}



\section{Preparação dos dados}

\subsection{Limpeza dos Dados}

\begin{tcolorbox}[breakable, size=fbox, boxrule=1pt, pad at break*=1mm,colback=cellbackground, colframe=cellborder]
    \prompt{In}{incolor}{8}{\boxspacing}
    \begin{Verbatim}[commandchars=\\\{\}]
        \PY{n+nf}{anyDuplicated}\PY{p}{(}\PY{n}{sleep\PYZus{}day\PYZus{}file}\PY{p}{)}
        \PY{n+nf}{anyDuplicated}\PY{p}{(}\PY{n}{daily\PYZus{}activity\PYZus{}file}\PY{p}{)}
        \PY{n+nf}{anyDuplicated}\PY{p}{(}\PY{n}{daily\PYZus{}intensities\PYZus{}file}\PY{p}{)}
        \PY{n+nf}{anyDuplicated}\PY{p}{(}\PY{n}{hourly\PYZus{}intensities\PYZus{}file}\PY{p}{)}
        \PY{n+nf}{anyDuplicated}\PY{p}{(}\PY{n}{hourly\PYZus{}calories\PYZus{}file}\PY{p}{)}
    \end{Verbatim}
\end{tcolorbox}

162


0


0


0


0


aqui podemos identificar que temos 162 duplicatas no sleep day file

\subsubsection{Descartando NA e duplicatas}

\begin{tcolorbox}[breakable, size=fbox, boxrule=1pt, pad at break*=1mm,colback=cellbackground, colframe=cellborder]
    \prompt{In}{incolor}{9}{\boxspacing}
    \begin{Verbatim}[commandchars=\\\{\}]
        \PY{n}{sleep\PYZus{}day\PYZus{}file}\PY{+w}{ }\PY{o}{\PYZlt{}\PYZhy{}}\PY{+w}{ }\PY{n}{sleep\PYZus{}day\PYZus{}file}\PY{+w}{ }\PY{o}{\PYZpc{}\PYZgt{}\PYZpc{}}
        \PY{+w}{  }\PY{n+nf}{distinct}\PY{p}{(}\PY{p}{)}\PY{+w}{ }\PY{o}{\PYZpc{}\PYZgt{}\PYZpc{}}
        \PY{+w}{  }\PY{n+nf}{drop\PYZus{}na}\PY{p}{(}\PY{p}{)}

        \PY{n+nf}{anyDuplicated}\PY{p}{(}\PY{n}{sleep\PYZus{}day\PYZus{}file}\PY{p}{)}
    \end{Verbatim}
\end{tcolorbox}

0


Tratamentos dos dados feitos, e agora temos 0 dados duplicados. Esse
tratamento é feito para evitar dados iguais esses dados duplicados podem
trazer vieses para as nossas analises futuras.

\subsubsection{Limpando os nomes para o formato usado nas aulas}

\begin{tcolorbox}[breakable, size=fbox, boxrule=1pt, pad at break*=1mm,colback=cellbackground, colframe=cellborder]
    \prompt{In}{incolor}{10}{\boxspacing}
    \begin{Verbatim}[commandchars=\\\{\}]
        \PY{n}{sleep\PYZus{}day\PYZus{}file}\PY{+w}{ }\PY{o}{\PYZlt{}\PYZhy{}}\PY{+w}{ }\PY{n+nf}{clean\PYZus{}names}\PY{p}{(}\PY{n}{sleep\PYZus{}day\PYZus{}file}\PY{p}{)}
        \PY{n}{daily\PYZus{}activity\PYZus{}file}\PY{+w}{ }\PY{o}{\PYZlt{}\PYZhy{}}\PY{+w}{ }\PY{n+nf}{clean\PYZus{}names}\PY{p}{(}\PY{n}{daily\PYZus{}activity\PYZus{}file}\PY{p}{)}
        \PY{n}{daily\PYZus{}intensities\PYZus{}file}\PY{+w}{ }\PY{o}{\PYZlt{}\PYZhy{}}\PY{+w}{ }\PY{n+nf}{clean\PYZus{}names}\PY{p}{(}\PY{n}{daily\PYZus{}intensities\PYZus{}file}\PY{p}{)}
        \PY{n}{hourly\PYZus{}intensities\PYZus{}file}\PY{+w}{ }\PY{o}{\PYZlt{}\PYZhy{}}\PY{+w}{ }\PY{n+nf}{clean\PYZus{}names}\PY{p}{(}\PY{n}{hourly\PYZus{}intensities\PYZus{}file}\PY{p}{)}
        \PY{n}{hourly\PYZus{}calories\PYZus{}file}\PY{+w}{ }\PY{o}{\PYZlt{}\PYZhy{}}\PY{+w}{ }\PY{n+nf}{clean\PYZus{}names}\PY{p}{(}\PY{n}{hourly\PYZus{}calories\PYZus{}file}\PY{p}{)}
    \end{Verbatim}
\end{tcolorbox}

agora temos os nossos nomes de colunas em um formato padronizado;
name1\_name2 (lowcase,separado por \_)

\subsubsection{Transformando as Datas}

\begin{tcolorbox}[breakable, size=fbox, boxrule=1pt, pad at break*=1mm,colback=cellbackground, colframe=cellborder]
    \prompt{In}{incolor}{11}{\boxspacing}
    \begin{Verbatim}[commandchars=\\\{\}]
        \PY{n+nf}{glimpse}\PY{p}{(}\PY{n}{sleep\PYZus{}day\PYZus{}file}\PY{p}{)}
        \PY{n+nf}{glimpse}\PY{p}{(}\PY{n}{daily\PYZus{}activity\PYZus{}file}\PY{p}{)}
        \PY{n+nf}{glimpse}\PY{p}{(}\PY{n}{daily\PYZus{}intensities\PYZus{}file}\PY{p}{)}
        \PY{n+nf}{glimpse}\PY{p}{(}\PY{n}{hourly\PYZus{}intensities\PYZus{}file}\PY{p}{)}
        \PY{n+nf}{glimpse}\PY{p}{(}\PY{n}{hourly\PYZus{}calories\PYZus{}file}\PY{p}{)}
    \end{Verbatim}
\end{tcolorbox}

\begin{Verbatim}[commandchars=\\\{\}]
    Rows: 410
    Columns: 5
    \$ id                   \textcolor{ansi-black-intense}{<dbl>} 1503960366, 1503960366,
    1503960366, 1503960366, 1…
    \$ sleep\_day            \textcolor{ansi-black-intense}{<chr>} "4/12/2016 12:00:00 AM",
    "4/13/2016 12:00:00 AM",…
    \$ total\_sleep\_records  \textcolor{ansi-black-intense}{<dbl>} 1, 2, 1, 2, 1, 1, 1, 1, 1, 1, 1,
    1, 1, 1, 1, 1, 1…
    \$ total\_minutes\_asleep \textcolor{ansi-black-intense}{<dbl>} 327, 384, 412, 340, 700, 304,
    360, 325, 361, 430,…
    \$ total\_time\_in\_bed    \textcolor{ansi-black-intense}{<dbl>} 346, 407, 442, 367, 712, 320,
    377, 364, 384, 449,…
    Rows: 940
    Columns: 15
    \$ id                         \textcolor{ansi-black-intense}{<dbl>} 1503960366, 1503960366,
    1503960366, 1503960…
    \$ activity\_date              \textcolor{ansi-black-intense}{<chr>} "4/12/2016", "4/13/2016",
    "4/14/2016", "4/1…
    \$ total\_steps                \textcolor{ansi-black-intense}{<dbl>} 13162, 10735, 10460, 9762,
    12669, 9705, 130…
    \$ total\_distance             \textcolor{ansi-black-intense}{<dbl>} 8.50, 6.97, 6.74, 6.28,
    8.16, 6.48, 8.59, 9…
    \$ tracker\_distance           \textcolor{ansi-black-intense}{<dbl>} 8.50, 6.97, 6.74, 6.28,
    8.16, 6.48, 8.59, 9…
    \$ logged\_activities\_distance \textcolor{ansi-black-intense}{<dbl>} 0, 0, 0, 0, 0, 0, 0, 0, 0,
    0, 0, 0, 0, 0, 0…
    \$ very\_active\_distance       \textcolor{ansi-black-intense}{<dbl>} 1.88, 1.57, 2.44, 2.14,
    2.71, 3.19, 3.25, 3…
    \$ moderately\_active\_distance \textcolor{ansi-black-intense}{<dbl>} 0.55, 0.69, 0.40, 1.26,
    0.41, 0.78, 0.64, 1…
    \$ light\_active\_distance      \textcolor{ansi-black-intense}{<dbl>} 6.06, 4.71, 3.91, 2.83,
    5.04, 2.51, 4.71, 5…
    \$ sedentary\_active\_distance  \textcolor{ansi-black-intense}{<dbl>} 0, 0, 0, 0, 0, 0, 0, 0, 0,
    0, 0, 0, 0, 0, 0…
    \$ very\_active\_minutes        \textcolor{ansi-black-intense}{<dbl>} 25, 21, 30, 29, 36, 38,
    42, 50, 28, 19, 66,…
    \$ fairly\_active\_minutes      \textcolor{ansi-black-intense}{<dbl>} 13, 19, 11, 34, 10, 20,
    16, 31, 12, 8, 27, …
    \$ lightly\_active\_minutes     \textcolor{ansi-black-intense}{<dbl>} 328, 217, 181, 209, 221,
    164, 233, 264, 205…
    \$ sedentary\_minutes          \textcolor{ansi-black-intense}{<dbl>} 728, 776, 1218, 726, 773,
    539, 1149, 775, 8…
    \$ calories                   \textcolor{ansi-black-intense}{<dbl>} 1985, 1797, 1776, 1745,
    1863, 1728, 1921, 2…
    Rows: 940
    Columns: 10
    \$ id                         \textcolor{ansi-black-intense}{<dbl>} 1503960366, 1503960366,
    1503960366, 1503960…
    \$ activity\_day               \textcolor{ansi-black-intense}{<chr>} "4/12/2016", "4/13/2016",
    "4/14/2016", "4/1…
    \$ sedentary\_minutes          \textcolor{ansi-black-intense}{<dbl>} 728, 776, 1218, 726, 773,
    539, 1149, 775, 8…
    \$ lightly\_active\_minutes     \textcolor{ansi-black-intense}{<dbl>} 328, 217, 181, 209, 221,
    164, 233, 264, 205…
    \$ fairly\_active\_minutes      \textcolor{ansi-black-intense}{<dbl>} 13, 19, 11, 34, 10, 20,
    16, 31, 12, 8, 27, …
    \$ very\_active\_minutes        \textcolor{ansi-black-intense}{<dbl>} 25, 21, 30, 29, 36, 38,
    42, 50, 28, 19, 66,…
    \$ sedentary\_active\_distance  \textcolor{ansi-black-intense}{<dbl>} 0, 0, 0, 0, 0, 0, 0, 0, 0,
    0, 0, 0, 0, 0, 0…
    \$ light\_active\_distance      \textcolor{ansi-black-intense}{<dbl>} 6.06, 4.71, 3.91, 2.83,
    5.04, 2.51, 4.71, 5…
    \$ moderately\_active\_distance \textcolor{ansi-black-intense}{<dbl>} 0.55, 0.69, 0.40, 1.26,
    0.41, 0.78, 0.64, 1…
    \$ very\_active\_distance       \textcolor{ansi-black-intense}{<dbl>} 1.88, 1.57, 2.44, 2.14,
    2.71, 3.19, 3.25, 3…
    Rows: 22,099
    Columns: 4
    \$ id                \textcolor{ansi-black-intense}{<dbl>} 1503960366, 1503960366, 1503960366,
    1503960366, 1503…
    \$ activity\_hour     \textcolor{ansi-black-intense}{<chr>} "4/12/2016 12:00:00 AM", "4/12/2016
    1:00:00 AM", "4/…
    \$ total\_intensity   \textcolor{ansi-black-intense}{<dbl>} 20, 8, 7, 0, 0, 0, 0, 0, 13, 30,
    29, 12, 11, 6, 36, …
    \$ average\_intensity \textcolor{ansi-black-intense}{<dbl>} 0.333333, 0.133333, 0.116667,
    0.000000, 0.000000, 0.…
    Rows: 22,099
    Columns: 3
    \$ id            \textcolor{ansi-black-intense}{<dbl>} 1503960366, 1503960366, 1503960366,
    1503960366, 15039603…
    \$ activity\_hour \textcolor{ansi-black-intense}{<chr>} "4/12/2016 12:00:00 AM", "4/12/2016
    1:00:00 AM", "4/12/2…
    \$ calories      \textcolor{ansi-black-intense}{<dbl>} 81, 61, 59, 47, 48, 48, 48, 47, 68,
    141, 99, 76, 73, 66,…
\end{Verbatim}

iremos fazer a transformação das colunas que tratam de \textbf{horas e
    datas} para facilicar no momento das nossas analises.

\begin{tcolorbox}[breakable, size=fbox, boxrule=1pt, pad at break*=1mm,colback=cellbackground, colframe=cellborder]
    \prompt{In}{incolor}{12}{\boxspacing}
    \begin{Verbatim}[commandchars=\\\{\}]
        \PY{n}{sleep\PYZus{}day\PYZus{}file}\PY{o}{\PYZdl{}}\PY{n}{sleep\PYZus{}day}\PY{+w}{ }\PY{o}{\PYZlt{}\PYZhy{}}\PY{+w}{ }\PY{n+nf}{mdy\PYZus{}hms}\PY{p}{(}\PY{n}{sleep\PYZus{}day\PYZus{}file}\PY{o}{\PYZdl{}}\PY{n}{sleep\PYZus{}day}\PY{p}{)}
        \PY{n}{daily\PYZus{}activity\PYZus{}file}\PY{o}{\PYZdl{}}\PY{n}{activity\PYZus{}date}\PY{+w}{ }\PY{o}{\PYZlt{}\PYZhy{}}\PY{+w}{ }\PY{n+nf}{mdy}\PY{p}{(}\PY{n}{daily\PYZus{}activity\PYZus{}file}\PY{o}{\PYZdl{}}\PY{n}{activity\PYZus{}date}\PY{p}{)}
        \PY{n}{daily\PYZus{}intensities\PYZus{}file}\PY{o}{\PYZdl{}}\PY{n}{activity\PYZus{}day}\PY{+w}{ }\PY{o}{\PYZlt{}\PYZhy{}}\PY{+w}{ }\PY{n+nf}{mdy}\PY{p}{(}\PY{n}{daily\PYZus{}intensities\PYZus{}file}\PY{o}{\PYZdl{}}\PY{n}{activity\PYZus{}day}\PY{p}{)}
        \PY{n}{hourly\PYZus{}intensities\PYZus{}file}\PY{o}{\PYZdl{}}\PY{n}{activity\PYZus{}hour}\PY{+w}{ }\PY{o}{\PYZlt{}\PYZhy{}}\PY{+w}{ }\PY{n+nf}{mdy\PYZus{}hms}\PY{p}{(}\PY{n}{hourly\PYZus{}intensities\PYZus{}file}\PY{o}{\PYZdl{}}\PY{n}{activity\PYZus{}hour}\PY{p}{)}
        \PY{n}{hourly\PYZus{}calories\PYZus{}file}\PY{o}{\PYZdl{}}\PY{n}{activity\PYZus{}hour}\PY{+w}{ }\PY{o}{\PYZlt{}\PYZhy{}}\PY{+w}{ }\PY{n+nf}{mdy\PYZus{}hms}\PY{p}{(}\PY{n}{hourly\PYZus{}calories\PYZus{}file}\PY{o}{\PYZdl{}}\PY{n}{activity\PYZus{}hour}\PY{p}{)}
    \end{Verbatim}
\end{tcolorbox}

A partir do \textbf{glimpse()} podemos identificar que as colunas
referentes a hora e datas agora estão nos seguintes formatos de dados:
\textbf{dttm,date}. Esse processamento é possível a partir da biblioteca
\textbf{\emph{lubridate}} que nos da as funções e argumentos para tratar
esse tipo de situações e muito mais.

\subsubsubsection{Transformações extras}

tratando dados para fazer o grafico de total de
\hyperref[subsection4-four]{Total de Passos diários Vs. Total de Minutos de Sono}

\begin{tcolorbox}[breakable, size=fbox, boxrule=1pt, pad at break*=1mm,colback=cellbackground, colframe=cellborder]
    \prompt{In}{incolor}{13}{\boxspacing}
    \begin{Verbatim}[commandchars=\\\{\}]
        \PY{n+nf}{glimpse}\PY{p}{(}\PY{n}{sleep\PYZus{}day\PYZus{}file}\PY{p}{)}
        \PY{n+nf}{glimpse}\PY{p}{(}\PY{n}{daily\PYZus{}activity\PYZus{}file}\PY{p}{)}
    \end{Verbatim}
\end{tcolorbox}

\begin{Verbatim}[commandchars=\\\{\}]
    Rows: 410
    Columns: 5
    \$ id                   \textcolor{ansi-black-intense}{<dbl>} 1503960366, 1503960366,
    1503960366, 1503960366, 1…
    \$ sleep\_day            \textcolor{ansi-black-intense}{<dttm>} 2016-04-12, 2016-04-13,
    2016-04-15, 2016-04-16, …
    \$ total\_sleep\_records  \textcolor{ansi-black-intense}{<dbl>} 1, 2, 1, 2, 1, 1, 1, 1, 1, 1, 1,
    1, 1, 1, 1, 1, 1…
    \$ total\_minutes\_asleep \textcolor{ansi-black-intense}{<dbl>} 327, 384, 412, 340, 700, 304,
    360, 325, 361, 430,…
    \$ total\_time\_in\_bed    \textcolor{ansi-black-intense}{<dbl>} 346, 407, 442, 367, 712, 320,
    377, 364, 384, 449,…
    Rows: 940
    Columns: 15
    \$ id                         \textcolor{ansi-black-intense}{<dbl>} 1503960366, 1503960366,
    1503960366, 1503960…
    \$ activity\_date              \textcolor{ansi-black-intense}{<date>} 2016-04-12, 2016-04-13,
    2016-04-14, 2016-0…
    \$ total\_steps                \textcolor{ansi-black-intense}{<dbl>} 13162, 10735, 10460, 9762,
    12669, 9705, 130…
    \$ total\_distance             \textcolor{ansi-black-intense}{<dbl>} 8.50, 6.97, 6.74, 6.28,
    8.16, 6.48, 8.59, 9…
    \$ tracker\_distance           \textcolor{ansi-black-intense}{<dbl>} 8.50, 6.97, 6.74, 6.28,
    8.16, 6.48, 8.59, 9…
    \$ logged\_activities\_distance \textcolor{ansi-black-intense}{<dbl>} 0, 0, 0, 0, 0, 0, 0, 0, 0,
    0, 0, 0, 0, 0, 0…
    \$ very\_active\_distance       \textcolor{ansi-black-intense}{<dbl>} 1.88, 1.57, 2.44, 2.14,
    2.71, 3.19, 3.25, 3…
    \$ moderately\_active\_distance \textcolor{ansi-black-intense}{<dbl>} 0.55, 0.69, 0.40, 1.26,
    0.41, 0.78, 0.64, 1…
    \$ light\_active\_distance      \textcolor{ansi-black-intense}{<dbl>} 6.06, 4.71, 3.91, 2.83,
    5.04, 2.51, 4.71, 5…
    \$ sedentary\_active\_distance  \textcolor{ansi-black-intense}{<dbl>} 0, 0, 0, 0, 0, 0, 0, 0, 0,
    0, 0, 0, 0, 0, 0…
    \$ very\_active\_minutes        \textcolor{ansi-black-intense}{<dbl>} 25, 21, 30, 29, 36, 38,
    42, 50, 28, 19, 66,…
    \$ fairly\_active\_minutes      \textcolor{ansi-black-intense}{<dbl>} 13, 19, 11, 34, 10, 20,
    16, 31, 12, 8, 27, …
    \$ lightly\_active\_minutes     \textcolor{ansi-black-intense}{<dbl>} 328, 217, 181, 209, 221,
    164, 233, 264, 205…
    \$ sedentary\_minutes          \textcolor{ansi-black-intense}{<dbl>} 728, 776, 1218, 726, 773,
    539, 1149, 775, 8…
    \$ calories                   \textcolor{ansi-black-intense}{<dbl>} 1985, 1797, 1776, 1745,
    1863, 1728, 1921, 2…
\end{Verbatim}

Para dar merge() os dados diários de sono e atividade com base em ID e
data, precisaremos padronizar os nomes das colunas de ID e Data.

Renomeando a coluna que exibe as datas para que possa ser usada como uma
das chaves durante a merge() de dados:

\begin{tcolorbox}[breakable, size=fbox, boxrule=1pt, pad at break*=1mm,colback=cellbackground, colframe=cellborder]
    \prompt{In}{incolor}{14}{\boxspacing}
    \begin{Verbatim}[commandchars=\\\{\}]
        \PY{n}{daily\PYZus{}activity\PYZus{}file}\PY{+w}{ }\PY{o}{\PYZlt{}\PYZhy{}}\PY{+w}{ }\PY{n}{daily\PYZus{}activity\PYZus{}file}\PY{+w}{ }\PY{o}{\PYZpc{}\PYZgt{}\PYZpc{}}\PY{+w}{ }\PY{n+nf}{rename}\PY{p}{(}\PY{n}{date}\PY{+w}{ }\PY{o}{=}\PY{+w}{ }\PY{n}{activity\PYZus{}date}\PY{p}{)}
        \PY{n}{sleep\PYZus{}day\PYZus{}file}\PY{+w}{ }\PY{o}{\PYZlt{}\PYZhy{}}\PY{+w}{ }\PY{n}{sleep\PYZus{}day\PYZus{}file}\PY{+w}{ }\PY{o}{\PYZpc{}\PYZgt{}\PYZpc{}}\PY{+w}{ }\PY{n+nf}{rename}\PY{p}{(}\PY{n}{date}\PY{+w}{ }\PY{o}{=}\PY{+w}{ }\PY{n}{sleep\PYZus{}day}\PY{p}{)}
    \end{Verbatim}
\end{tcolorbox}

Merge() os subconjuntos `sleep' e `activity' para criar o conjunto de
dados `activity\_sleep', usando ``id'' e ``date'' como chaves:

\begin{tcolorbox}[breakable, size=fbox, boxrule=1pt, pad at break*=1mm,colback=cellbackground, colframe=cellborder]
    \prompt{In}{incolor}{15}{\boxspacing}
    \begin{Verbatim}[commandchars=\\\{\}]
        \PY{n}{activity\PYZus{}sleep}\PY{+w}{ }\PY{o}{\PYZlt{}\PYZhy{}}\PY{+w}{ }\PY{n+nf}{merge}\PY{p}{(}\PY{n}{daily\PYZus{}activity\PYZus{}file}\PY{p}{,}\PY{+w}{ }\PY{n}{sleep\PYZus{}day\PYZus{}file}\PY{p}{,}\PY{+w}{ }\PY{n}{by}\PY{o}{=}\PY{n+nf}{c}\PY{p}{(}\PY{l+s}{\PYZdq{}}\PY{l+s}{id\PYZdq{}}\PY{p}{,}\PY{+w}{ }\PY{l+s}{\PYZdq{}}\PY{l+s}{date\PYZdq{}}\PY{p}{)}\PY{p}{)}


        \PY{n+nf}{glimpse}\PY{p}{(}\PY{n}{activity\PYZus{}sleep}\PY{p}{)}
    \end{Verbatim}
\end{tcolorbox}

\begin{Verbatim}[commandchars=\\\{\}]
    Rows: 410
    Columns: 18
    \$ id                         \textcolor{ansi-black-intense}{<dbl>} 1503960366, 1503960366,
    1503960366, 1503960…
    \$ date                       \textcolor{ansi-black-intense}{<date>} 2016-04-12, 2016-04-13,
    2016-04-15, 2016-0…
    \$ total\_steps                \textcolor{ansi-black-intense}{<dbl>} 13162, 10735, 9762, 12669,
    9705, 15506, 105…
    \$ total\_distance             \textcolor{ansi-black-intense}{<dbl>} 8.50, 6.97, 6.28, 8.16,
    6.48, 9.88, 6.68, 6…
    \$ tracker\_distance           \textcolor{ansi-black-intense}{<dbl>} 8.50, 6.97, 6.28, 8.16,
    6.48, 9.88, 6.68, 6…
    \$ logged\_activities\_distance \textcolor{ansi-black-intense}{<dbl>} 0, 0, 0, 0, 0, 0, 0, 0, 0,
    0, 0, 0, 0, 0, 0…
    \$ very\_active\_distance       \textcolor{ansi-black-intense}{<dbl>} 1.88, 1.57, 2.14, 2.71,
    3.19, 3.53, 1.96, 1…
    \$ moderately\_active\_distance \textcolor{ansi-black-intense}{<dbl>} 0.55, 0.69, 1.26, 0.41,
    0.78, 1.32, 0.48, 0…
    \$ light\_active\_distance      \textcolor{ansi-black-intense}{<dbl>} 6.06, 4.71, 2.83, 5.04,
    2.51, 5.03, 4.24, 4…
    \$ sedentary\_active\_distance  \textcolor{ansi-black-intense}{<dbl>} 0, 0, 0, 0, 0, 0, 0, 0, 0,
    0, 0, 0, 0, 0, 0…
    \$ very\_active\_minutes        \textcolor{ansi-black-intense}{<dbl>} 25, 21, 29, 36, 38, 50,
    28, 19, 41, 39, 73,…
    \$ fairly\_active\_minutes      \textcolor{ansi-black-intense}{<dbl>} 13, 19, 34, 10, 20, 31,
    12, 8, 21, 5, 14, 2…
    \$ lightly\_active\_minutes     \textcolor{ansi-black-intense}{<dbl>} 328, 217, 209, 221, 164,
    264, 205, 211, 262…
    \$ sedentary\_minutes          \textcolor{ansi-black-intense}{<dbl>} 728, 776, 726, 773, 539,
    775, 818, 838, 732…
    \$ calories                   \textcolor{ansi-black-intense}{<dbl>} 1985, 1797, 1745, 1863,
    1728, 2035, 1786, 1…
    \$ total\_sleep\_records        \textcolor{ansi-black-intense}{<dbl>} 1, 2, 1, 2, 1, 1, 1, 1, 1,
    1, 1, 1, 1, 1, 1…
    \$ total\_minutes\_asleep       \textcolor{ansi-black-intense}{<dbl>} 327, 384, 412, 340, 700,
    304, 360, 325, 361…
    \$ total\_time\_in\_bed          \textcolor{ansi-black-intense}{<dbl>} 346, 407, 442, 367, 712,
    320, 377, 364, 384…
\end{Verbatim}

\textbf{Abaixo} trataremos os dados para o
\hyperref[subsection4-three]{Grafico - Calorias x intensidade}. Vamos
colocar as datas em dia para conseguiremos agrupar por esse conjunto de
dados, e vamos dar um \textbf{cbind()} nos valores de calorias essa
função pega uma sequência de argumentos de vetor, matriz ou quadro de
dados e combine por colunas ou linhas, respectivamente.

\begin{tcolorbox}[breakable, size=fbox, boxrule=1pt, pad at break*=1mm,colback=cellbackground, colframe=cellborder]
    \prompt{In}{incolor}{16}{\boxspacing}
    \begin{Verbatim}[commandchars=\\\{\}]
        \PY{n}{hourly\PYZus{}intensities\PYZus{}file}\PY{o}{\PYZdl{}}\PY{n}{day}\PY{+w}{ }\PY{o}{\PYZlt{}\PYZhy{}}\PY{+w}{ }\PY{n+nf}{format}\PY{p}{(}\PY{n}{hourly\PYZus{}intensities\PYZus{}file}\PY{o}{\PYZdl{}}\PY{n}{activity\PYZus{}hour}\PY{p}{,}\PY{+w}{ }\PY{n}{format}\PY{+w}{ }\PY{o}{=}\PY{+w}{ }\PY{l+s}{\PYZdq{}}\PY{l+s}{\PYZpc{}Y \PYZpc{}m \PYZpc{}d\PYZdq{}}\PY{p}{)}
        \PY{n}{hourly\PYZus{}intensities\PYZus{}file}\PY{o}{\PYZdl{}}\PY{n}{calories}\PY{+w}{ }\PY{o}{\PYZlt{}\PYZhy{}}\PY{+w}{ }\PY{n+nf}{cbind}\PY{p}{(}\PY{n}{hourly\PYZus{}calories\PYZus{}file}\PY{o}{\PYZdl{}}\PY{n}{calories}\PY{p}{)}
        \PY{n+nf}{glimpse}\PY{p}{(}\PY{n}{hourly\PYZus{}intensities\PYZus{}file}\PY{p}{)}
    \end{Verbatim}
\end{tcolorbox}

\begin{Verbatim}[commandchars=\\\{\}]
    Rows: 22,099
    Columns: 6
    \$ id                \textcolor{ansi-black-intense}{<dbl>} 1503960366, 1503960366, 1503960366,
    1503960366, 1503…
    \$ activity\_hour     \textcolor{ansi-black-intense}{<dttm>} 2016-04-12 00:00:00, 2016-04-12
    01:00:00, 2016-04-1…
    \$ total\_intensity   \textcolor{ansi-black-intense}{<dbl>} 20, 8, 7, 0, 0, 0, 0, 0, 13, 30,
    29, 12, 11, 6, 36, …
    \$ average\_intensity \textcolor{ansi-black-intense}{<dbl>} 0.333333, 0.133333, 0.116667,
    0.000000, 0.000000, 0.…
    \$ day               \textcolor{ansi-black-intense}{<chr>} "2016 04 12", "2016 04 12", "2016
    04 12", "2016 04 1…
    \$ calories          \textcolor{ansi-black-intense}{<dbl[,1]>} <matrix[26 x 1]>
\end{Verbatim}

\textbf{Abaixo} adicionaremos dias da semana aos conjuntos de dados para
auxiliar nas análises refentes ao gráfico
\hyperref[subsection4-five]{Tempo de sono}

\begin{tcolorbox}[breakable, size=fbox, boxrule=1pt, pad at break*=1mm,colback=cellbackground, colframe=cellborder]
    \prompt{In}{incolor}{17}{\boxspacing}
    \begin{Verbatim}[commandchars=\\\{\}]
        \PY{n}{sleep\PYZus{}day\PYZus{}file}\PY{+w}{ }\PY{o}{\PYZlt{}\PYZhy{}}\PY{+w}{ }\PY{n}{sleep\PYZus{}day\PYZus{}file}\PY{+w}{ }\PY{o}{\PYZpc{}\PYZgt{}\PYZpc{}}\PY{+w}{ }\PY{n+nf}{mutate}\PY{p}{(}\PY{n}{day\PYZus{}of\PYZus{}week}\PY{+w}{ }\PY{o}{=}\PY{+w}{ }\PY{n+nf}{weekdays}\PY{p}{(}\PY{n}{date}\PY{p}{)}\PY{p}{)}
        \PY{c+c1}{\PYZsh{} sleep\PYZus{}day\PYZus{}file\PYZdl{}weekday \PYZlt{}\PYZhy{} weekdays(sleep\PYZus{}day\PYZus{}file\PYZdl{}sleep\PYZus{}day) }
        \PY{n}{sleep\PYZus{}day\PYZus{}file}\PY{o}{\PYZdl{}}\PY{n}{day\PYZus{}of\PYZus{}week}\PY{+w}{ }\PY{o}{\PYZlt{}\PYZhy{}}\PY{+w}{ }\PY{n+nf}{factor}\PY{p}{(}\PY{n}{sleep\PYZus{}day\PYZus{}file}\PY{o}{\PYZdl{}}\PY{n}{day\PYZus{}of\PYZus{}week}\PY{p}{,}\PY{n}{levels}\PY{+w}{ }\PY{o}{=}\PY{+w}{ }\PY{n+nf}{c}\PY{p}{(}\PY{l+s}{\PYZdq{}}\PY{l+s}{Monday\PYZdq{}}\PY{p}{,}\PY{+w}{ }\PY{l+s}{\PYZdq{}}\PY{l+s}{Tuesday\PYZdq{}}\PY{p}{,}\PY{+w}{ }\PY{l+s}{\PYZdq{}}\PY{l+s}{Wednesday\PYZdq{}}\PY{p}{,}\PY{+w}{ }\PY{l+s}{\PYZdq{}}\PY{l+s}{Thursday\PYZdq{}}\PY{p}{,}\PY{+w}{ }\PY{l+s}{\PYZdq{}}\PY{l+s}{Friday\PYZdq{}}\PY{p}{,}\PY{+w}{ }\PY{l+s}{\PYZdq{}}\PY{l+s}{Saturday\PYZdq{}}\PY{p}{,}\PY{+w}{ }\PY{l+s}{\PYZdq{}}\PY{l+s}{Sunday\PYZdq{}}\PY{p}{)}\PY{p}{)}
    \end{Verbatim}
\end{tcolorbox}

Aqui fizemos um mutate para adicionar a coluna referente a dias da
semana. Passamos \emph{date} para a \emph{função} weekdays para assim
termos os dias da semana refentes as data do nosso dataframe. Em
seguida, fazemos um \emph{factor} para que os dias da semana sejam
organizadas da ordem passada no vetor \emph{c} em \emph{levels}.

\begin{tcolorbox}[breakable, size=fbox, boxrule=1pt, pad at break*=1mm,colback=cellbackground, colframe=cellborder]
    \prompt{In}{incolor}{18}{\boxspacing}
    \begin{Verbatim}[commandchars=\\\{\}]
        \PY{n+nf}{unique}\PY{p}{(}\PY{n}{sleep\PYZus{}day\PYZus{}file}\PY{o}{\PYZdl{}}\PY{n}{day\PYZus{}of\PYZus{}week}\PY{p}{)}
    \end{Verbatim}
\end{tcolorbox}

\begin{enumerate*}
    \item Tuesday
    \item Wednesday
    \item Friday
    \item Saturday
    \item Sunday
    \item Thursday
    \item Monday
\end{enumerate*}

\emph{Levels}: \begin{enumerate*}
    \item 'Monday'
    \item 'Tuesday'
    \item 'Wednesday'
    \item 'Thursday'
    \item 'Friday'
    \item 'Saturday'
    \item 'Sunday'
\end{enumerate*}



testando para ver se temos os dias referentes a todos os dias da semana

\textbf{Abaixo} vamos fazer o somatório para o gráfico
\hyperref[subsection4-six]{Uso diário dos dispositivos} que nós
apresenta a quantidade de horas que os nossos usuários passam usando os
aparelhos inteligentes.

\begin{tcolorbox}[breakable, size=fbox, boxrule=1pt, pad at break*=1mm,colback=cellbackground, colframe=cellborder]
    \prompt{In}{incolor}{19}{\boxspacing}
    \begin{Verbatim}[commandchars=\\\{\}]
        \PY{n}{daily\PYZus{}activity\PYZus{}file}\PY{o}{\PYZdl{}}\PY{n}{total\PYZus{}time}\PY{+w}{ }\PY{o}{=}\PY{+w}{ }\PY{n+nf}{rowSums}\PY{p}{(}\PY{n}{daily\PYZus{}activity\PYZus{}file}\PY{p}{[}\PY{n+nf}{c}\PY{p}{(}\PY{l+s}{\PYZdq{}}\PY{l+s}{very\PYZus{}active\PYZus{}minutes\PYZdq{}}\PY{p}{,}
        \PY{+w}{                                                               }\PY{l+s}{\PYZdq{}}\PY{l+s}{fairly\PYZus{}active\PYZus{}minutes\PYZdq{}}\PY{p}{,}
        \PY{+w}{                                                               }\PY{l+s}{\PYZdq{}}\PY{l+s}{lightly\PYZus{}active\PYZus{}minutes\PYZdq{}}\PY{p}{,}
        \PY{+w}{                                                               }\PY{l+s}{\PYZdq{}}\PY{l+s}{sedentary\PYZus{}minutes\PYZdq{}}\PY{p}{)}\PY{p}{]}\PY{p}{)}
        \PY{n+nf}{glimpse}\PY{p}{(}\PY{n}{daily\PYZus{}activity\PYZus{}file}\PY{p}{)}
    \end{Verbatim}
\end{tcolorbox}

\begin{Verbatim}[commandchars=\\\{\}]
    Rows: 940
    Columns: 16
    \$ id                         \textcolor{ansi-black-intense}{<dbl>} 1503960366, 1503960366,
    1503960366, 1503960…
    \$ date                       \textcolor{ansi-black-intense}{<date>} 2016-04-12, 2016-04-13,
    2016-04-14, 2016-0…
    \$ total\_steps                \textcolor{ansi-black-intense}{<dbl>} 13162, 10735, 10460, 9762,
    12669, 9705, 130…
    \$ total\_distance             \textcolor{ansi-black-intense}{<dbl>} 8.50, 6.97, 6.74, 6.28,
    8.16, 6.48, 8.59, 9…
    \$ tracker\_distance           \textcolor{ansi-black-intense}{<dbl>} 8.50, 6.97, 6.74, 6.28,
    8.16, 6.48, 8.59, 9…
    \$ logged\_activities\_distance \textcolor{ansi-black-intense}{<dbl>} 0, 0, 0, 0, 0, 0, 0, 0, 0,
    0, 0, 0, 0, 0, 0…
    \$ very\_active\_distance       \textcolor{ansi-black-intense}{<dbl>} 1.88, 1.57, 2.44, 2.14,
    2.71, 3.19, 3.25, 3…
    \$ moderately\_active\_distance \textcolor{ansi-black-intense}{<dbl>} 0.55, 0.69, 0.40, 1.26,
    0.41, 0.78, 0.64, 1…
    \$ light\_active\_distance      \textcolor{ansi-black-intense}{<dbl>} 6.06, 4.71, 3.91, 2.83,
    5.04, 2.51, 4.71, 5…
    \$ sedentary\_active\_distance  \textcolor{ansi-black-intense}{<dbl>} 0, 0, 0, 0, 0, 0, 0, 0, 0,
    0, 0, 0, 0, 0, 0…
    \$ very\_active\_minutes        \textcolor{ansi-black-intense}{<dbl>} 25, 21, 30, 29, 36, 38,
    42, 50, 28, 19, 66,…
    \$ fairly\_active\_minutes      \textcolor{ansi-black-intense}{<dbl>} 13, 19, 11, 34, 10, 20,
    16, 31, 12, 8, 27, …
    \$ lightly\_active\_minutes     \textcolor{ansi-black-intense}{<dbl>} 328, 217, 181, 209, 221,
    164, 233, 264, 205…
    \$ sedentary\_minutes          \textcolor{ansi-black-intense}{<dbl>} 728, 776, 1218, 726, 773,
    539, 1149, 775, 8…
    \$ calories                   \textcolor{ansi-black-intense}{<dbl>} 1985, 1797, 1776, 1745,
    1863, 1728, 1921, 2…
    \$ total\_time                 \textcolor{ansi-black-intense}{<dbl>} 1094, 1033, 1440, 998,
    1040, 761, 1440, 112…
\end{Verbatim}

Fizemos o somatório das colunas \emph{very\_active\_minutes},
\emph{fairly\_active\_minutes}, \emph{lightly\_active\_minutes},
\emph{sedentary\_minutes} usando a função \textbf{rowSums()}

\section{Análise dos dados}

\subsection{Analisando}

Nessa parte do projeto iremos dar nossas primeira olhada nos gráficos
possiveis para o nosso BD.

\subsubsection{Calorias x Passos}

Agora vamos dar uma olhada nas calorias gastas pelos usuários e fazer
uns graficos na tentativa de buscar alguma compreensão do que está
acontecendo.

\begin{tcolorbox}[breakable, size=fbox, boxrule=1pt, pad at break*=1mm,colback=cellbackground, colframe=cellborder]
    \prompt{In}{incolor}{20}{\boxspacing}
    \begin{Verbatim}[commandchars=\\\{\}]
        \PY{n}{daily\PYZus{}activity\PYZus{}file}\PY{+w}{ }\PY{o}{\PYZpc{}\PYZgt{}\PYZpc{}}
        \PY{+w}{  }\PY{n+nf}{ggplot}\PY{p}{(}\PY{p}{)}\PY{+w}{ }\PY{o}{+}
        \PY{+w}{  }\PY{p}{(}\PY{n}{mapping}\PY{+w}{ }\PY{o}{=}\PY{+w}{ }\PY{n+nf}{aes}\PY{p}{(}\PY{n}{x}\PY{+w}{ }\PY{o}{=}\PY{+w}{ }\PY{n}{total\PYZus{}steps}\PY{p}{,}\PY{+w}{ }\PY{n}{y}\PY{+w}{ }\PY{o}{=}\PY{+w}{ }\PY{n}{calories}\PY{p}{)}\PY{p}{)}\PY{+w}{ }\PY{o}{+}
        \PY{+w}{  }\PY{n+nf}{geom\PYZus{}jitter}\PY{p}{(}\PY{p}{)}\PY{+w}{ }\PY{o}{+}
        \PY{+w}{  }\PY{n+nf}{geom\PYZus{}smooth}\PY{p}{(}\PY{p}{)}\PY{+w}{ }\PY{o}{+}
        \PY{+w}{  }\PY{n+nf}{stat\PYZus{}cor}\PY{p}{(}\PY{n}{method}\PY{+w}{ }\PY{o}{=}\PY{+w}{ }\PY{l+s}{\PYZdq{}}\PY{l+s}{pearson\PYZdq{}}\PY{p}{,}\PY{+w}{ }\PY{n}{label.x}\PY{+w}{ }\PY{o}{=}\PY{+w}{ }\PY{l+m}{20000}\PY{p}{,}\PY{+w}{ }\PY{n}{label.y}\PY{+w}{ }\PY{o}{=}\PY{+w}{ }\PY{l+m}{4800}\PY{p}{)}\PY{+w}{ }\PY{o}{+}
        \PY{+w}{  }\PY{n+nf}{scale\PYZus{}color\PYZus{}igv}\PY{p}{(}\PY{p}{)}\PY{+w}{ }\PY{o}{+}
        \PY{+w}{  }\PY{n+nf}{scale\PYZus{}fill\PYZus{}igv}\PY{p}{(}\PY{p}{)}\PY{+w}{ }\PY{o}{+}
        \PY{+w}{  }\PY{n+nf}{theme\PYZus{}grey}\PY{p}{(}\PY{p}{)}\PY{+w}{ }\PY{o}{+}
        \PY{+w}{  }\PY{n+nf}{labs}\PY{p}{(}
        \PY{+w}{    }\PY{n}{title}\PY{+w}{ }\PY{o}{=}\PY{+w}{ }\PY{l+s}{\PYZdq{}}\PY{l+s}{Passos diários vs. calorias\PYZdq{}}\PY{p}{,}
        \PY{+w}{    }\PY{n}{subtitle}\PY{+w}{ }\PY{o}{=}\PY{+w}{ }\PY{l+s}{\PYZdq{}}\PY{l+s}{Coeficiente de correlação de Pearson (r)\PYZdq{}}\PY{p}{,}
        \PY{+w}{    }\PY{n}{x}\PY{+w}{ }\PY{o}{=}\PY{+w}{ }\PY{l+s}{\PYZdq{}}\PY{l+s}{Passos diários\PYZdq{}}\PY{p}{,}
        \PY{+w}{    }\PY{n}{y}\PY{+w}{ }\PY{o}{=}\PY{+w}{ }\PY{l+s}{\PYZdq{}}\PY{l+s}{Calorias Queimadas\PYZdq{}}
        \PY{+w}{  }\PY{p}{)}
    \end{Verbatim}
\end{tcolorbox}

\begin{Verbatim}[commandchars=\\\{\}]
    `geom\_smooth()` using method = 'loess' and formula = 'y \textasciitilde{} x'
\end{Verbatim}

\begin{center}
    \adjustimage{max size={0.9\linewidth}{0.9\paperheight}}{finalproject_files/finalproject_53_1.png}
\end{center}
{ \hspace*{\fill} \\}

A escala do coeficiente de
\href{https://www.scribbr.com/statistics/pearson-correlation-coefficient/}{Correlação
    de Pearson} (r = 0,59)
\href{https://www.researchgate.net/figure/The-scale-of-Pearsons-Correlation-Coefficient_tbl1_345693737}{(correlação
    moderada)}, entre 0 e 1 é uma correlação positiva. Podemos identificar
que temos uma correlação moderada isso nos diz que outros fatores podem
estar afetando/colaborando para tal projeção do nosso conjunto de dados,
fatores como: passos mais rápidos dos usuários\ldots{}

Vamos dar uma olhada na intensidade x calorias para identificamos se
haverá uma correlação entre esses dados.

\subsection{Calorias x Intensidade}

\begin{tcolorbox}[breakable, size=fbox, boxrule=1pt, pad at break*=1mm,colback=cellbackground, colframe=cellborder]
    \prompt{In}{incolor}{21}{\boxspacing}
    \begin{Verbatim}[commandchars=\\\{\}]
        \PY{n}{graph\PYZus{}intensidade\PYZus{}calorias}\PY{+w}{ }\PY{o}{\PYZlt{}\PYZhy{}}\PY{+w}{ }\PY{n}{hourly\PYZus{}intensities\PYZus{}file}\PY{+w}{ }\PY{o}{\PYZpc{}\PYZgt{}\PYZpc{}}
        \PY{+w}{  }\PY{n+nf}{group\PYZus{}by}\PY{p}{(}\PY{n}{day}\PY{p}{)}\PY{+w}{ }\PY{o}{\PYZpc{}\PYZgt{}\PYZpc{}}
        \PY{+w}{  }\PY{n+nf}{summarise}\PY{p}{(}\PY{n}{total\PYZus{}int}\PY{+w}{ }\PY{o}{=}\PY{+w}{ }\PY{n}{total\PYZus{}intensity}\PY{p}{,}\PY{+w}{ }\PY{n}{total\PYZus{}cal}\PY{+w}{ }\PY{o}{=}\PY{+w}{ }\PY{n}{calories}\PY{p}{)}\PY{+w}{ }\PY{o}{\PYZpc{}\PYZgt{}\PYZpc{}}
        \PY{+w}{  }\PY{n+nf}{ggplot}\PY{p}{(}\PY{p}{)}\PY{+w}{ }\PY{o}{+}
        \PY{+w}{  }\PY{p}{(}\PY{n}{mapping}\PY{+w}{ }\PY{o}{=}\PY{+w}{ }\PY{n+nf}{aes}\PY{p}{(}\PY{n}{x}\PY{+w}{ }\PY{o}{=}\PY{+w}{ }\PY{n}{total\PYZus{}int}\PY{p}{,}\PY{+w}{ }\PY{n}{y}\PY{+w}{ }\PY{o}{=}\PY{+w}{ }\PY{n}{total\PYZus{}cal}\PY{p}{)}\PY{p}{)}\PY{+w}{ }\PY{o}{+}
        \PY{+w}{  }\PY{n+nf}{geom\PYZus{}jitter}\PY{p}{(}\PY{p}{)}\PY{+w}{ }\PY{o}{+}
        \PY{+w}{  }\PY{n+nf}{geom\PYZus{}smooth}\PY{p}{(}\PY{p}{)}\PY{+w}{ }\PY{o}{+}
        \PY{+w}{  }\PY{n+nf}{stat\PYZus{}cor}\PY{p}{(}\PY{n}{method}\PY{+w}{ }\PY{o}{=}\PY{+w}{ }\PY{l+s}{\PYZdq{}}\PY{l+s}{pearson\PYZdq{}}\PY{p}{,}\PY{+w}{ }\PY{n}{label.x}\PY{+w}{ }\PY{o}{=}\PY{+w}{ }\PY{l+m}{50}\PY{p}{,}\PY{+w}{ }\PY{n}{label.y}\PY{+w}{ }\PY{o}{=}\PY{+w}{ }\PY{l+m}{750}\PY{p}{)}\PY{+w}{ }\PY{o}{+}
        \PY{+w}{  }\PY{n+nf}{labs}\PY{p}{(}
        \PY{+w}{    }\PY{n}{title}\PY{+w}{ }\PY{o}{=}\PY{+w}{ }\PY{l+s}{\PYZdq{}}\PY{l+s}{Horárias de Intensidade vs. Calorias\PYZdq{}}\PY{p}{,}
        \PY{+w}{    }\PY{n}{x}\PY{+w}{ }\PY{o}{=}\PY{+w}{ }\PY{l+s}{\PYZdq{}}\PY{l+s}{Horárias de Intensidade\PYZdq{}}\PY{p}{,}
        \PY{+w}{    }\PY{n}{y}\PY{+w}{ }\PY{o}{=}\PY{+w}{ }\PY{l+s}{\PYZdq{}}\PY{l+s}{Calorias\PYZdq{}}
        \PY{+w}{  }\PY{p}{)}
        \PY{n}{graph\PYZus{}intensidade\PYZus{}calorias}
    \end{Verbatim}
\end{tcolorbox}

\begin{Verbatim}[commandchars=\\\{\}]
    `summarise()` has grouped output by 'day'. You can override using the
    `.groups`
    argument.
    `geom\_smooth()` using method = 'gam' and formula = 'y \textasciitilde{} s(x, bs =
    "cs")'
\end{Verbatim}

\begin{center}
    \adjustimage{max size={0.9\linewidth}{0.9\paperheight}}{finalproject_files/finalproject_57_1.png}
\end{center}
{ \hspace*{\fill} \\}

Agora podemos notar uma correlação muito alta; (R = 0,9). Quando uma
variável muda, a outra variável muda na mesma direção.

\subsection{Total de Passos diários Vs. Total de Minutos de Sono}

\begin{tcolorbox}[breakable, size=fbox, boxrule=1pt, pad at break*=1mm,colback=cellbackground, colframe=cellborder]
    \prompt{In}{incolor}{22}{\boxspacing}
    \begin{Verbatim}[commandchars=\\\{\}]
        \PY{n}{graph\PYZus{}passos\PYZus{}sono}\PY{+w}{ }\PY{o}{\PYZlt{}\PYZhy{}}\PY{+w}{ }\PY{n}{activity\PYZus{}sleep}\PY{+w}{ }\PY{o}{\PYZpc{}\PYZgt{}\PYZpc{}}\PY{+w}{ }
        \PY{+w}{    }\PY{n+nf}{ggplot}\PY{p}{(}
        \PY{+w}{     }\PY{n}{mapping}\PY{+w}{ }\PY{o}{=}\PY{+w}{ }\PY{n+nf}{aes}\PY{p}{(}\PY{n}{x}\PY{+w}{ }\PY{o}{=}\PY{+w}{ }\PY{n}{total\PYZus{}steps}\PY{p}{,}\PY{+w}{ }\PY{n}{y}\PY{+w}{ }\PY{o}{=}\PY{+w}{ }\PY{n}{total\PYZus{}minutes\PYZus{}asleep}\PY{o}{/}\PY{l+m}{60}\PY{p}{)}\PY{p}{)}\PY{+w}{ }\PY{o}{+}
        \PY{+w}{      }\PY{n+nf}{scale\PYZus{}y\PYZus{}continuous}\PY{p}{(}\PY{n}{breaks}\PY{+w}{ }\PY{o}{=}\PY{+w}{ }\PY{n+nf}{c}\PY{p}{(}\PY{l+m}{0}\PY{o}{:}\PY{l+m}{24}\PY{p}{)}\PY{p}{)}\PY{+w}{ }\PY{o}{+}
        \PY{+w}{      }\PY{n+nf}{geom\PYZus{}point}\PY{p}{(}\PY{p}{)}\PY{+w}{ }\PY{o}{+}
        \PY{+w}{      }\PY{n+nf}{geom\PYZus{}smooth}\PY{p}{(}\PY{p}{)}\PY{+w}{ }\PY{o}{+}
        \PY{+w}{      }\PY{n+nf}{geom\PYZus{}jitter}\PY{p}{(}\PY{p}{)}\PY{+w}{ }\PY{o}{+}
        \PY{+w}{      }\PY{n+nf}{geom\PYZus{}smooth}\PY{p}{(}\PY{p}{)}\PY{+w}{ }\PY{o}{+}
        \PY{+w}{      }\PY{n+nf}{stat\PYZus{}cor}\PY{p}{(}\PY{n}{method}\PY{+w}{ }\PY{o}{=}\PY{+w}{ }\PY{l+s}{\PYZdq{}}\PY{l+s}{pearson\PYZdq{}}\PY{p}{,}\PY{+w}{ }\PY{n}{label.x}\PY{+w}{ }\PY{o}{=}\PY{+w}{ }\PY{l+m}{16000}\PY{p}{,}\PY{+w}{ }\PY{n}{label.y}\PY{+w}{ }\PY{o}{=}\PY{+w}{ }\PY{l+m}{8}\PY{p}{)}\PY{+w}{ }\PY{o}{+}
        \PY{+w}{      }\PY{n+nf}{scale\PYZus{}color\PYZus{}igv}\PY{p}{(}\PY{p}{)}\PY{+w}{ }\PY{o}{+}
        \PY{+w}{      }\PY{n+nf}{scale\PYZus{}fill\PYZus{}igv}\PY{p}{(}\PY{p}{)}\PY{+w}{ }\PY{o}{+}
        \PY{+w}{      }\PY{n+nf}{theme\PYZus{}grey}\PY{p}{(}\PY{p}{)}\PY{+w}{ }\PY{o}{+}
        \PY{+w}{      }\PY{n+nf}{labs}\PY{p}{(}
        \PY{+w}{        }\PY{n}{title}\PY{+w}{ }\PY{o}{=}\PY{+w}{ }\PY{l+s}{\PYZdq{}}\PY{l+s}{Passos diários Vs. Horas de Sono\PYZdq{}}\PY{p}{,}
        \PY{+w}{        }\PY{n}{subtitle}\PY{+w}{ }\PY{o}{=}\PY{+w}{ }\PY{l+s}{\PYZdq{}}\PY{l+s}{Coeficiente de locomoção de Pearson (r)\PYZdq{}}\PY{p}{,}
        \PY{+w}{        }\PY{n}{x}\PY{+w}{ }\PY{o}{=}\PY{+w}{ }\PY{l+s}{\PYZdq{}}\PY{l+s}{Passos diários\PYZdq{}}\PY{p}{,}
        \PY{+w}{        }\PY{n}{y}\PY{+w}{ }\PY{o}{=}\PY{+w}{ }\PY{l+s}{\PYZdq{}}\PY{l+s}{Horas Dormindo\PYZdq{}}
        \PY{+w}{      }\PY{p}{)}
        \PY{n}{graph\PYZus{}passos\PYZus{}sono}
    \end{Verbatim}
\end{tcolorbox}

\begin{Verbatim}[commandchars=\\\{\}]
    `geom\_smooth()` using method = 'loess' and formula = 'y \textasciitilde{} x'
    `geom\_smooth()` using method = 'loess' and formula = 'y \textasciitilde{} x'
\end{Verbatim}

\begin{center}
    \adjustimage{max size={0.9\linewidth}{0.9\paperheight}}{finalproject_files/finalproject_60_1.png}
\end{center}
{ \hspace*{\fill} \\}

Diferentemente do gráfico de passos e calorias queimadas, em que quanto
mais passos forem dados mais calorias podem ser queimadas.

Neste gráfico de passos e a quantidade de minutos que os usuários dormem
por dia não há correlação entre as variáveis.

\subsection{Tempo de sono}

\begin{tcolorbox}[breakable, size=fbox, boxrule=1pt, pad at break*=1mm,colback=cellbackground, colframe=cellborder]
    \prompt{In}{incolor}{23}{\boxspacing}
    \begin{Verbatim}[commandchars=\\\{\}]
        \PY{n}{graph\PYZus{}sono\PYZus{}medio}\PY{+w}{ }\PY{o}{\PYZlt{}\PYZhy{}}\PY{+w}{ }\PY{n}{sleep\PYZus{}day\PYZus{}file}\PY{+w}{ }\PY{o}{\PYZpc{}\PYZgt{}\PYZpc{}}\PY{+w}{ }
        \PY{+w}{  }\PY{n+nf}{group\PYZus{}by}\PY{p}{(}\PY{n}{day\PYZus{}of\PYZus{}week}\PY{p}{)}\PY{+w}{ }\PY{o}{\PYZpc{}\PYZgt{}\PYZpc{}}\PY{+w}{ }
        \PY{+w}{  }\PY{n+nf}{summarise}\PY{p}{(}\PY{n}{avg\PYZus{}sleep}\PY{+w}{ }\PY{o}{=}\PY{+w}{ }\PY{n+nf}{mean}\PY{p}{(}\PY{n}{total\PYZus{}minutes\PYZus{}asleep}\PY{+w}{ }\PY{p}{)}\PY{o}{/}\PY{l+m}{60}\PY{p}{)}\PY{+w}{ }\PY{o}{\PYZpc{}\PYZgt{}\PYZpc{}}\PY{+w}{ }
        \PY{+w}{  }\PY{n+nf}{ggplot}\PY{p}{(}\PY{p}{)}\PY{+w}{ }\PY{o}{+}
        \PY{+w}{  }\PY{n+nf}{scale\PYZus{}y\PYZus{}continuous}\PY{p}{(}\PY{n}{breaks}\PY{+w}{ }\PY{o}{=}\PY{+w}{ }\PY{n+nf}{c}\PY{p}{(}\PY{l+m}{0}\PY{o}{:}\PY{l+m}{24}\PY{p}{)}\PY{p}{)}\PY{+w}{ }\PY{o}{+}
        \PY{+w}{  }\PY{n+nf}{geom\PYZus{}col}\PY{p}{(}\PY{n}{mapping}\PY{+w}{ }\PY{o}{=}\PY{+w}{ }\PY{n+nf}{aes}\PY{p}{(}\PY{n}{x}\PY{o}{=}\PY{+w}{ }\PY{n}{day\PYZus{}of\PYZus{}week}\PY{p}{,}\PY{+w}{ }\PY{n}{y}\PY{+w}{ }\PY{o}{=}\PY{+w}{ }\PY{n}{avg\PYZus{}sleep}\PY{p}{)}\PY{p}{,}\PY{+w}{ }\PY{n}{fill}\PY{+w}{ }\PY{o}{=}\PY{+w}{ }\PY{l+s}{\PYZdq{}}\PY{l+s}{darkgreen\PYZdq{}}\PY{p}{)}\PY{+w}{ }\PY{o}{+}
        \PY{+w}{  }\PY{n+nf}{geom\PYZus{}hline}\PY{p}{(}\PY{n+nf}{aes}\PY{p}{(}\PY{n}{yintercept}\PY{+w}{ }\PY{o}{=}\PY{+w}{ }\PY{l+m}{7}\PY{p}{)}\PY{p}{)}\PY{+w}{ }\PY{o}{+}
        \PY{+w}{  }\PY{n+nf}{annotate}\PY{p}{(}\PY{n}{geom}\PY{o}{=}\PY{l+s}{\PYZdq{}}\PY{l+s}{text\PYZdq{}}\PY{p}{,}\PY{+w}{ }\PY{n}{x}\PY{o}{=}\PY{l+m}{2.5}\PY{p}{,}\PY{+w}{ }\PY{n}{y}\PY{o}{=}\PY{+w}{ }\PY{l+m}{7.5}\PY{p}{,}\PY{+w}{ }\PY{n}{label}\PY{o}{=}\PY{l+s}{\PYZdq{}}\PY{l+s}{7 horas de sono necessárias\PYZdq{}}\PY{p}{,}
        \PY{+w}{           }\PY{n}{color}\PY{o}{=}\PY{l+s}{\PYZdq{}}\PY{l+s}{darkgreen\PYZdq{}}\PY{p}{)}\PY{+w}{ }\PY{o}{+}
        \PY{+w}{  }\PY{n+nf}{theme}\PY{p}{(}\PY{n}{axis.text.x}\PY{+w}{ }\PY{o}{=}\PY{+w}{ }\PY{n+nf}{element\PYZus{}text}\PY{p}{(}\PY{n}{angle}\PY{+w}{ }\PY{o}{=}\PY{+w}{ }\PY{l+m}{90}\PY{p}{)}\PY{p}{)}\PY{+w}{ }\PY{o}{+}
        \PY{+w}{  }\PY{n+nf}{labs}\PY{p}{(}\PY{n}{title}\PY{o}{=}\PY{l+s}{\PYZdq{}}\PY{l+s}{Tempo médio de sono\PYZdq{}}\PY{p}{,}
        \PY{+w}{       }\PY{n}{y}\PY{o}{=}\PY{l+s}{\PYZdq{}}\PY{l+s}{Horas\PYZdq{}}\PY{p}{,}
        \PY{+w}{      }\PY{n}{x}\PY{+w}{ }\PY{o}{=}\PY{+w}{ }\PY{l+s}{\PYZdq{}}\PY{l+s}{Dias da semana\PYZdq{}}\PY{p}{)}
        \PY{n}{graph\PYZus{}sono\PYZus{}medio}
    \end{Verbatim}
\end{tcolorbox}

\begin{center}
    \adjustimage{max size={0.9\linewidth}{0.9\paperheight}}{finalproject_files/finalproject_63_0.png}
\end{center}
{ \hspace*{\fill} \\}

Parece que os usuários nem sempre dormem pelo menos 7 horas por dia.
Sabendo que nossos usuários são adultos devido a necessidade de
consentimento para a cessão dos dados. Podemos induzir que esses
usuários precisariam de 7 ou mais horas por noite, por estarem no
parâmetro 18--60 anos (adultos) como sugerido pela CDC.

Última revisão: 14 de setembro de 2022 Fonte:
\href{https://www.cdc.gov/sleep/about_sleep/how_much_sleep.html}{National
    Center for Chronic Disease Prevention and Health Promotion, Division of
    Population Health}

\subsubsection{Apenas brincando com os gráficos}

\subsubsubsection{Uso diário dos dispositivos}

Vamos dar uma espiada de como esses dados se comportam em um gráfico.

\begin{tcolorbox}[breakable, size=fbox, boxrule=1pt, pad at break*=1mm,colback=cellbackground, colframe=cellborder]
    \prompt{In}{incolor}{24}{\boxspacing}
    \begin{Verbatim}[commandchars=\\\{\}]
        \PY{n}{daily\PYZus{}activity\PYZus{}file}\PY{+w}{ }\PY{o}{\PYZpc{}\PYZgt{}\PYZpc{}}
        \PY{+w}{  }\PY{n+nf}{group\PYZus{}by}\PY{p}{(}\PY{n}{id}\PY{p}{)}\PY{+w}{ }\PY{o}{\PYZpc{}\PYZgt{}\PYZpc{}}
        \PY{+w}{  }\PY{n+nf}{summarise}\PY{p}{(}\PY{n}{daily\PYZus{}usage\PYZus{}hour}\PY{+w}{ }\PY{o}{=}\PY{+w}{ }\PY{n+nf}{mean}\PY{p}{(}\PY{n}{total\PYZus{}time}\PY{+w}{ }\PY{o}{/}\PY{+w}{ }\PY{l+m}{60}\PY{p}{)}\PY{p}{)}\PY{+w}{ }\PY{o}{\PYZpc{}\PYZgt{}\PYZpc{}}
        \PY{+w}{  }\PY{n+nf}{ggplot}\PY{p}{(}\PY{n+nf}{aes}\PY{p}{(}\PY{n}{x}\PY{+w}{ }\PY{o}{=}\PY{+w}{ }\PY{n}{daily\PYZus{}usage\PYZus{}hour}\PY{p}{)}\PY{p}{)}\PY{+w}{ }\PY{o}{+}
        \PY{+w}{  }\PY{n+nf}{geom\PYZus{}histogram}\PY{p}{(}
        \PY{+w}{    }\PY{n}{color}\PY{+w}{ }\PY{o}{=}\PY{+w}{ }\PY{l+s}{\PYZdq{}}\PY{l+s}{black\PYZdq{}}\PY{p}{,}\PY{+w}{ }\PY{n}{fill}\PY{+w}{ }\PY{o}{=}\PY{+w}{ }\PY{l+s}{\PYZdq{}}\PY{l+s}{\PYZsh{}008b3a\PYZdq{}}
        \PY{+w}{  }\PY{p}{)}\PY{+w}{ }\PY{o}{+}
        \PY{+w}{  }\PY{n+nf}{scale\PYZus{}color\PYZus{}igv}\PY{p}{(}\PY{p}{)}\PY{+w}{ }\PY{o}{+}
        \PY{+w}{  }\PY{n+nf}{scale\PYZus{}fill\PYZus{}igv}\PY{p}{(}\PY{p}{)}\PY{+w}{ }\PY{o}{+}
        \PY{+w}{  }\PY{n+nf}{theme\PYZus{}grey}\PY{p}{(}\PY{p}{)}\PY{+w}{ }\PY{o}{+}
        \PY{+w}{  }\PY{n+nf}{scale\PYZus{}x\PYZus{}continuous}\PY{p}{(}\PY{n}{breaks}\PY{+w}{ }\PY{o}{=}\PY{+w}{ }\PY{n+nf}{c}\PY{p}{(}\PY{l+m}{1}\PY{o}{:}\PY{l+m}{24}\PY{p}{)}\PY{p}{)}\PY{+w}{ }\PY{o}{+}
        \PY{+w}{  }\PY{n+nf}{labs}\PY{p}{(}
        \PY{+w}{    }\PY{n}{title}\PY{+w}{ }\PY{o}{=}\PY{+w}{ }\PY{l+s}{\PYZdq{}}\PY{l+s}{Tempo médio diário de uso do dispositivo\PYZdq{}}\PY{p}{,}
        \PY{+w}{    }\PY{n}{subtitle}\PY{+w}{ }\PY{o}{=}\PY{+w}{ }\PY{l+s}{\PYZdq{}}\PY{l+s}{Escala só irá conter a media do total de tempo de uso\PYZdq{}}\PY{p}{,}
        \PY{+w}{    }\PY{n}{x}\PY{+w}{ }\PY{o}{=}\PY{+w}{ }\PY{l+s}{\PYZdq{}}\PY{l+s}{Tempo de uso diário (horas)\PYZdq{}}\PY{p}{,}
        \PY{+w}{    }\PY{n}{y}\PY{+w}{ }\PY{o}{=}\PY{+w}{ }\PY{l+s}{\PYZdq{}}\PY{l+s}{Contagem\PYZdq{}}
        \PY{+w}{  }\PY{p}{)}
    \end{Verbatim}
\end{tcolorbox}

\begin{Verbatim}[commandchars=\\\{\}]
    `stat\_bin()` using `bins = 30`. Pick better value with `binwidth`.
\end{Verbatim}

\begin{center}
    \adjustimage{max size={0.9\linewidth}{0.9\paperheight}}{finalproject_files/finalproject_68_1.png}
\end{center}
{ \hspace*{\fill} \\}

podemos identificar no gráfico a media de tempo que os usuarios passam
usando o dispositivo inteligente.

\emph{foi usado so pipes antifos \%\textgreater\% pois o kaggle nao
    estava identificando os novos \textbar\textgreater{}}

\subsubsubsection{Multi gráficos}

\begin{tcolorbox}[breakable, size=fbox, boxrule=1pt, pad at break*=1mm,colback=cellbackground, colframe=cellborder]
    \prompt{In}{incolor}{25}{\boxspacing}
    \begin{Verbatim}[commandchars=\\\{\}]
        \PY{n}{curr\PYZus{}locale}\PY{+w}{ }\PY{o}{\PYZlt{}\PYZhy{}}\PY{+w}{ }\PY{n+nf}{Sys.getlocale}\PY{p}{(}\PY{l+s}{\PYZdq{}}\PY{l+s}{LC\PYZus{}TIME\PYZdq{}}\PY{p}{)}
        \PY{n+nf}{Sys.setlocale}\PY{p}{(}\PY{l+s}{\PYZdq{}}\PY{l+s}{LC\PYZus{}TIME\PYZdq{}}\PY{p}{,}\PY{l+s}{\PYZdq{}}\PY{l+s}{pt\PYZus{}BR\PYZdq{}}\PY{p}{)}

        \PY{c+c1}{\PYZsh{}\PYZlt{}specific code for graph generation\PYZgt{}}

        \PY{c+c1}{\PYZsh{}Sys.setlocale(\PYZdq{}LC\PYZus{}TIME\PYZdq{},curr\PYZus{}locale)}
    \end{Verbatim}
\end{tcolorbox}

\begin{Verbatim}[commandchars=\\\{\}]
    Warning message in Sys.setlocale("LC\_TIME", "pt\_BR"):
    “OS reports request to set locale to "pt\_BR" cannot be honored”
\end{Verbatim}

''


tentativa de mudar a localização para fazer uso da função weekdays em
português. Entretanto, não foi possível.

\begin{tcolorbox}[breakable, size=fbox, boxrule=1pt, pad at break*=1mm,colback=cellbackground, colframe=cellborder]
    \prompt{In}{incolor}{26}{\boxspacing}
    \begin{Verbatim}[commandchars=\\\{\}]
        \PY{n}{weekday\PYZus{}steps\PYZus{}sleep}\PY{+w}{ }\PY{o}{\PYZlt{}\PYZhy{}}\PY{+w}{ }\PY{n}{activity\PYZus{}sleep}\PY{+w}{ }\PY{o}{\PYZpc{}\PYZgt{}\PYZpc{}}
        \PY{+w}{  }\PY{n+nf}{mutate}\PY{p}{(}\PY{n}{day\PYZus{}of\PYZus{}week}\PY{+w}{ }\PY{o}{=}\PY{+w}{ }\PY{n+nf}{weekdays}\PY{p}{(}\PY{n}{date}\PY{p}{)}\PY{p}{)}


        \PY{n}{weekday\PYZus{}steps\PYZus{}sleep}\PY{o}{\PYZdl{}}\PY{n}{day\PYZus{}of\PYZus{}week}\PY{+w}{ }\PY{o}{\PYZlt{}\PYZhy{}}\PY{n+nf}{ordered}\PY{p}{(}\PY{n}{weekday\PYZus{}steps\PYZus{}sleep}\PY{o}{\PYZdl{}}\PY{n}{day\PYZus{}of\PYZus{}week}\PY{p}{,}\PY{+w}{ }\PY{n}{levels}\PY{o}{=}\PY{n+nf}{c}\PY{p}{(}\PY{l+s}{\PYZdq{}}\PY{l+s}{Monday\PYZdq{}}\PY{p}{,}\PY{+w}{ }\PY{l+s}{\PYZdq{}}\PY{l+s}{Tuesday\PYZdq{}}\PY{p}{,}\PY{+w}{ }\PY{l+s}{\PYZdq{}}\PY{l+s}{Wednesday\PYZdq{}}\PY{p}{,}\PY{+w}{ }\PY{l+s}{\PYZdq{}}\PY{l+s}{Thursday\PYZdq{}}\PY{p}{,}
        \PY{l+s}{\PYZdq{}}\PY{l+s}{Friday\PYZdq{}}\PY{p}{,}\PY{+w}{ }\PY{l+s}{\PYZdq{}}\PY{l+s}{Saturday\PYZdq{}}\PY{p}{,}\PY{+w}{ }\PY{l+s}{\PYZdq{}}\PY{l+s}{Sunday\PYZdq{}}\PY{p}{)}\PY{p}{)}

        \PY{+w}{ }\PY{n}{weekday\PYZus{}steps\PYZus{}sleep}\PY{+w}{ }\PY{o}{\PYZlt{}\PYZhy{}}\PY{+w}{ }\PY{n}{weekday\PYZus{}steps\PYZus{}sleep}\PY{+w}{ }\PY{o}{\PYZpc{}\PYZgt{}\PYZpc{}}
        \PY{+w}{  }\PY{n+nf}{group\PYZus{}by}\PY{p}{(}\PY{n}{day\PYZus{}of\PYZus{}week}\PY{p}{)}\PY{+w}{ }\PY{o}{\PYZpc{}\PYZgt{}\PYZpc{}}
        \PY{+w}{  }\PY{n+nf}{summarize }\PY{p}{(}\PY{n}{daily\PYZus{}steps}\PY{+w}{ }\PY{o}{=}\PY{+w}{ }\PY{n+nf}{mean}\PY{p}{(}\PY{n}{total\PYZus{}steps}\PY{p}{)}\PY{p}{,}\PY{+w}{ }\PY{n}{daily\PYZus{}sleep\PYZus{}hour}\PY{+w}{ }\PY{o}{=}\PY{+w}{ }\PY{n+nf}{mean}\PY{p}{(}\PY{n}{total\PYZus{}minutes\PYZus{}asleep}\PY{p}{)}\PY{o}{/}\PY{l+m}{60}\PY{p}{)}

        \PY{n+nf}{head}\PY{p}{(}\PY{n}{weekday\PYZus{}steps\PYZus{}sleep}\PY{p}{)}

        \PY{n}{weekday\PYZus{}steps\PYZus{}sleep}\PY{o}{\PYZdl{}}\PY{n}{day\PYZus{}of\PYZus{}week}\PY{+w}{ }\PY{o}{\PYZlt{}\PYZhy{}}\PY{+w}{ }\PY{n+nf}{c}\PY{p}{(}\PY{l+s}{\PYZdq{}}\PY{l+s}{Segunda\PYZdq{}}\PY{p}{,}\PY{+w}{ }\PY{l+s}{\PYZdq{}}\PY{l+s}{Terca\PYZdq{}}\PY{p}{,}\PY{+w}{ }\PY{l+s}{\PYZdq{}}\PY{l+s}{Quarta\PYZdq{}}\PY{p}{,}\PY{+w}{ }\PY{l+s}{\PYZdq{}}\PY{l+s}{Quinta\PYZdq{}}\PY{p}{,}
        \PY{l+s}{\PYZdq{}}\PY{l+s}{Sexta\PYZdq{}}\PY{p}{,}\PY{+w}{ }\PY{l+s}{\PYZdq{}}\PY{l+s}{Sabado\PYZdq{}}\PY{p}{,}\PY{+w}{ }\PY{l+s}{\PYZdq{}}\PY{l+s}{Domingo\PYZdq{}}\PY{p}{)}
        \PY{n}{weekday\PYZus{}steps\PYZus{}sleep}\PY{o}{\PYZdl{}}\PY{n}{day\PYZus{}of\PYZus{}week}\PY{+w}{ }\PY{o}{\PYZlt{}\PYZhy{}}\PY{n+nf}{ordered}\PY{p}{(}\PY{n}{weekday\PYZus{}steps\PYZus{}sleep}\PY{o}{\PYZdl{}}\PY{n}{day\PYZus{}of\PYZus{}week}\PY{p}{,}\PY{+w}{ }\PY{n}{levels}\PY{o}{=}\PY{n+nf}{c}\PY{p}{(}\PY{l+s}{\PYZdq{}}\PY{l+s}{Segunda\PYZdq{}}\PY{p}{,}\PY{+w}{ }\PY{l+s}{\PYZdq{}}\PY{l+s}{Terca\PYZdq{}}\PY{p}{,}\PY{+w}{ }\PY{l+s}{\PYZdq{}}\PY{l+s}{Quarta\PYZdq{}}\PY{p}{,}\PY{+w}{ }\PY{l+s}{\PYZdq{}}\PY{l+s}{Quinta\PYZdq{}}\PY{p}{,}
        \PY{l+s}{\PYZdq{}}\PY{l+s}{Sexta\PYZdq{}}\PY{p}{,}\PY{+w}{ }\PY{l+s}{\PYZdq{}}\PY{l+s}{Sabado\PYZdq{}}\PY{p}{,}\PY{+w}{ }\PY{l+s}{\PYZdq{}}\PY{l+s}{Domingo\PYZdq{}}\PY{p}{)}\PY{p}{)}


        \PY{n+nf}{head}\PY{p}{(}\PY{n}{weekday\PYZus{}steps\PYZus{}sleep}\PY{p}{)}
    \end{Verbatim}
\end{tcolorbox}

A tibble: 6 × 3
\begin{tabular}{lll}
    day\_of\_week & daily\_steps & daily\_sleep\_hour \\
    <ord>         & <dbl>        & <dbl>              \\
    \hline
    Monday        & 9273.217     & 6.991667           \\
    Tuesday       & 9182.692     & 6.742308           \\
    Wednesday     & 8022.864     & 7.244697           \\
    Thursday      & 8183.516     & 6.688281           \\
    Friday        & 7901.404     & 6.757018           \\
    Saturday      & 9871.123     & 6.984503           \\
\end{tabular}



A tibble: 6 × 3
\begin{tabular}{lll}
    day\_of\_week & daily\_steps & daily\_sleep\_hour \\
    <ord>         & <dbl>        & <dbl>              \\
    \hline
    Segunda       & 9273.217     & 6.991667           \\
    Terca         & 9182.692     & 6.742308           \\
    Quarta        & 8022.864     & 7.244697           \\
    Quinta        & 8183.516     & 6.688281           \\
    Sexta         & 7901.404     & 6.757018           \\
    Sabado        & 9871.123     & 6.984503           \\
\end{tabular}



\begin{tcolorbox}[breakable, size=fbox, boxrule=1pt, pad at break*=1mm,colback=cellbackground, colframe=cellborder]
    \prompt{In}{incolor}{27}{\boxspacing}
    \begin{Verbatim}[commandchars=\\\{\}]
        \PY{n+nf}{ggarrange}\PY{p}{(}
        \PY{+w}{    }
        \PY{+w}{    }\PY{n+nf}{ggplot}\PY{p}{(}\PY{n}{weekday\PYZus{}steps\PYZus{}sleep}\PY{p}{)}\PY{+w}{ }\PY{o}{+}
        \PY{+w}{      }\PY{n+nf}{geom\PYZus{}col}\PY{p}{(}\PY{n+nf}{aes}\PY{p}{(}\PY{n}{day\PYZus{}of\PYZus{}week}\PY{p}{,}\PY{+w}{ }\PY{n}{daily\PYZus{}steps}\PY{p}{)}\PY{p}{,}\PY{+w}{ }\PY{n}{fill}\PY{+w}{ }\PY{o}{=}\PY{+w}{ }\PY{l+s}{\PYZdq{}}\PY{l+s}{\PYZsh{}006699\PYZdq{}}\PY{p}{)}\PY{+w}{ }\PY{o}{+}
        \PY{+w}{      }\PY{n+nf}{geom\PYZus{}hline}\PY{p}{(}\PY{n}{yintercept}\PY{+w}{ }\PY{o}{=}\PY{+w}{ }\PY{l+m}{7500}\PY{p}{)}\PY{+w}{ }\PY{o}{+}
        \PY{+w}{      }\PY{n+nf}{labs}\PY{p}{(}\PY{n}{title}\PY{+w}{ }\PY{o}{=}\PY{+w}{ }\PY{l+s}{\PYZdq{}}\PY{l+s}{Passos diários por dia da semana\PYZdq{}}\PY{p}{,}\PY{+w}{ }\PY{n}{x}\PY{o}{=}\PY{+w}{ }\PY{l+s}{\PYZdq{}}\PY{l+s}{\PYZdq{}}\PY{p}{,}\PY{+w}{ }\PY{n}{y}\PY{+w}{ }\PY{o}{=}\PY{+w}{ }\PY{l+s}{\PYZdq{}}\PY{l+s}{\PYZdq{}}\PY{p}{)}\PY{+w}{ }\PY{o}{+}
        \PY{+w}{      }\PY{n+nf}{theme}\PY{p}{(}\PY{n}{axis.text.x}\PY{+w}{ }\PY{o}{=}\PY{+w}{ }\PY{n+nf}{element\PYZus{}text}\PY{p}{(}\PY{n}{angle}\PY{+w}{ }\PY{o}{=}\PY{+w}{ }\PY{l+m}{45}\PY{p}{,}\PY{n}{vjust}\PY{+w}{ }\PY{o}{=}\PY{+w}{ }\PY{l+m}{0.5}\PY{p}{,}\PY{+w}{ }\PY{n}{hjust}\PY{+w}{ }\PY{o}{=}\PY{+w}{ }\PY{l+m}{1}\PY{p}{)}\PY{p}{)}\PY{p}{,}
        \PY{+w}{    }
        \PY{+w}{    }\PY{n+nf}{ggplot}\PY{p}{(}\PY{n}{weekday\PYZus{}steps\PYZus{}sleep}\PY{p}{,}\PY{+w}{ }\PY{n+nf}{aes}\PY{p}{(}\PY{n}{day\PYZus{}of\PYZus{}week}\PY{p}{,}\PY{+w}{ }\PY{n}{daily\PYZus{}sleep\PYZus{}hour}\PY{p}{)}\PY{p}{)}\PY{+w}{ }\PY{o}{+}
        \PY{+w}{      }\PY{n+nf}{geom\PYZus{}col}\PY{p}{(}\PY{n}{fill}\PY{+w}{ }\PY{o}{=}\PY{+w}{ }\PY{l+s}{\PYZdq{}}\PY{l+s}{\PYZsh{}85e0e0\PYZdq{}}\PY{p}{)}\PY{+w}{ }\PY{o}{+}
        \PY{+w}{      }\PY{n+nf}{scale\PYZus{}y\PYZus{}continuous}\PY{p}{(}\PY{n}{breaks}\PY{+w}{ }\PY{o}{=}\PY{+w}{ }\PY{n+nf}{c}\PY{p}{(}\PY{l+m}{0}\PY{o}{:}\PY{l+m}{24}\PY{p}{)}\PY{p}{)}\PY{+w}{ }\PY{o}{+}
        \PY{+w}{      }\PY{n+nf}{geom\PYZus{}hline}\PY{p}{(}\PY{n}{yintercept}\PY{+w}{ }\PY{o}{=}\PY{+w}{ }\PY{l+m}{7}\PY{p}{)}\PY{+w}{ }\PY{o}{+}
        \PY{+w}{      }\PY{n+nf}{labs}\PY{p}{(}\PY{n}{title}\PY{+w}{ }\PY{o}{=}\PY{+w}{ }\PY{l+s}{\PYZdq{}}\PY{l+s}{Horas de sono por dia da semana\PYZdq{}}\PY{p}{,}\PY{+w}{ }\PY{n}{x}\PY{o}{=}\PY{+w}{ }\PY{l+s}{\PYZdq{}}\PY{l+s}{\PYZdq{}}\PY{p}{,}\PY{+w}{ }\PY{n}{y}\PY{+w}{ }\PY{o}{=}\PY{+w}{ }\PY{l+s}{\PYZdq{}}\PY{l+s}{\PYZdq{}}\PY{p}{)}\PY{+w}{ }\PY{o}{+}
        \PY{+w}{      }\PY{n+nf}{theme}\PY{p}{(}\PY{n}{axis.text.x}\PY{+w}{ }\PY{o}{=}\PY{+w}{ }\PY{n+nf}{element\PYZus{}text}\PY{p}{(}\PY{n}{angle}\PY{+w}{ }\PY{o}{=}\PY{+w}{ }\PY{l+m}{45}\PY{p}{,}\PY{n}{vjust}\PY{+w}{ }\PY{o}{=}\PY{+w}{ }\PY{l+m}{0.5}\PY{p}{,}\PY{+w}{ }\PY{n}{hjust}\PY{+w}{ }\PY{o}{=}\PY{+w}{ }\PY{l+m}{1}\PY{p}{)}\PY{p}{)}
        \PY{+w}{  }\PY{p}{)}\PY{+w}{ }\PY{o}{\PYZhy{}\PYZgt{}}\PY{+w}{ }\PY{n}{bar\PYZus{}weekday\PYZus{}steps\PYZus{}sleep}

        \PY{n}{bar\PYZus{}weekday\PYZus{}steps\PYZus{}sleep}
    \end{Verbatim}
\end{tcolorbox}

\begin{center}
    \adjustimage{max size={0.9\linewidth}{0.9\paperheight}}{finalproject_files/finalproject_74_0.png}
\end{center}
{ \hspace*{\fill} \\}

\begin{tcolorbox}[breakable, size=fbox, boxrule=1pt, pad at break*=1mm,colback=cellbackground, colframe=cellborder]
    \prompt{In}{incolor}{28}{\boxspacing}
    \begin{Verbatim}[commandchars=\\\{\}]
        \PY{n+nf}{max}\PY{p}{(}\PY{n}{activity\PYZus{}sleep}\PY{o}{\PYZdl{}}\PY{n}{total\PYZus{}minutes\PYZus{}asleep}\PY{o}{/}\PY{l+m}{60}\PY{p}{)}
    \end{Verbatim}
\end{tcolorbox}

13.2666666666667


\begin{tcolorbox}[breakable, size=fbox, boxrule=1pt, pad at break*=1mm,colback=cellbackground, colframe=cellborder]
    \prompt{In}{incolor}{29}{\boxspacing}
    \begin{Verbatim}[commandchars=\\\{\}]
        \PY{n+nf}{ggarrange}\PY{p}{(}
        \PY{n+nf}{ggplot}\PY{p}{(}\PY{n}{activity\PYZus{}sleep}\PY{p}{,}\PY{+w}{ }\PY{n+nf}{aes}\PY{p}{(}\PY{n}{x}\PY{o}{=}\PY{n}{total\PYZus{}steps}\PY{p}{,}\PY{+w}{ }\PY{n}{y}\PY{o}{=}\PY{n}{total\PYZus{}minutes\PYZus{}asleep}\PY{o}{/}\PY{l+m}{60}\PY{p}{)}\PY{p}{)}\PY{o}{+}
        \PY{+w}{  }\PY{n+nf}{scale\PYZus{}y\PYZus{}continuous}\PY{p}{(}\PY{n}{breaks}\PY{+w}{ }\PY{o}{=}\PY{+w}{ }\PY{n+nf}{c}\PY{p}{(}\PY{l+m}{0}\PY{o}{:}\PY{l+m}{24}\PY{p}{)}\PY{p}{)}\PY{+w}{ }\PY{o}{+}
        \PY{+w}{  }\PY{n+nf}{geom\PYZus{}jitter}\PY{p}{(}\PY{p}{)}\PY{+w}{ }\PY{o}{+}
        \PY{+w}{  }\PY{n+nf}{geom\PYZus{}smooth}\PY{p}{(}\PY{n}{color}\PY{+w}{ }\PY{o}{=}\PY{+w}{ }\PY{l+s}{\PYZdq{}}\PY{l+s}{red\PYZdq{}}\PY{p}{)}\PY{+w}{ }\PY{o}{+}\PY{+w}{ }
        \PY{+w}{  }\PY{n+nf}{labs}\PY{p}{(}\PY{n}{title}\PY{+w}{ }\PY{o}{=}\PY{+w}{ }\PY{l+s}{\PYZdq{}}\PY{l+s}{Passos diários vs Horas dormindo\PYZdq{}}\PY{p}{,}
        \PY{+w}{       }\PY{n}{x}\PY{+w}{ }\PY{o}{=}\PY{+w}{ }\PY{l+s}{\PYZdq{}}\PY{l+s}{Passos diários\PYZdq{}}\PY{p}{,}
        \PY{+w}{       }\PY{n}{y}\PY{o}{=}\PY{+w}{ }\PY{l+s}{\PYZdq{}}\PY{l+s}{Horas dormindo\PYZdq{}}\PY{p}{)}\PY{+w}{ }\PY{o}{+}
        \PY{+w}{   }\PY{n+nf}{theme}\PY{p}{(}\PY{n}{panel.background}\PY{+w}{ }\PY{o}{=}\PY{+w}{ }\PY{n+nf}{element\PYZus{}blank}\PY{p}{(}\PY{p}{)}\PY{p}{,}
        \PY{+w}{        }\PY{n}{plot.title}\PY{+w}{ }\PY{o}{=}\PY{+w}{ }\PY{n+nf}{element\PYZus{}text}\PY{p}{(}\PY{+w}{ }\PY{n}{size}\PY{o}{=}\PY{l+m}{14}\PY{p}{)}\PY{p}{)}\PY{p}{,}\PY{+w}{ }
        \PY{n+nf}{ggplot}\PY{p}{(}\PY{n}{activity\PYZus{}sleep}\PY{p}{,}\PY{+w}{ }\PY{n+nf}{aes}\PY{p}{(}\PY{n}{x}\PY{o}{=}\PY{n}{total\PYZus{}steps}\PY{p}{,}\PY{+w}{ }\PY{n}{y}\PY{o}{=}\PY{n}{calories}\PY{p}{)}\PY{p}{)}\PY{o}{+}
        \PY{+w}{  }\PY{n+nf}{geom\PYZus{}jitter}\PY{p}{(}\PY{p}{)}\PY{+w}{ }\PY{o}{+}
        \PY{+w}{  }\PY{n+nf}{geom\PYZus{}smooth}\PY{p}{(}\PY{n}{color}\PY{+w}{ }\PY{o}{=}\PY{+w}{ }\PY{l+s}{\PYZdq{}}\PY{l+s}{red\PYZdq{}}\PY{p}{)}\PY{+w}{ }\PY{o}{+}\PY{+w}{ }
        \PY{+w}{  }\PY{n+nf}{labs}\PY{p}{(}\PY{n}{title}\PY{+w}{ }\PY{o}{=}\PY{+w}{ }\PY{l+s}{\PYZdq{}}\PY{l+s}{Passos diários vs Calorias\PYZdq{}}\PY{p}{,}
        \PY{+w}{       }\PY{n}{x}\PY{+w}{ }\PY{o}{=}\PY{+w}{ }\PY{l+s}{\PYZdq{}}\PY{l+s}{Passos diários\PYZdq{}}\PY{p}{,}
        \PY{+w}{       }\PY{n}{y}\PY{o}{=}\PY{+w}{ }\PY{l+s}{\PYZdq{}}\PY{l+s}{Calorias\PYZdq{}}\PY{p}{)}\PY{+w}{ }\PY{o}{+}
        \PY{+w}{   }\PY{n+nf}{theme}\PY{p}{(}\PY{n}{panel.background}\PY{+w}{ }\PY{o}{=}\PY{+w}{ }\PY{n+nf}{element\PYZus{}blank}\PY{p}{(}\PY{p}{)}\PY{p}{,}
        \PY{+w}{        }\PY{n}{plot.title}\PY{+w}{ }\PY{o}{=}\PY{+w}{ }\PY{n+nf}{element\PYZus{}text}\PY{p}{(}\PY{+w}{ }\PY{n}{size}\PY{o}{=}\PY{l+m}{14}\PY{p}{)}\PY{p}{)}
        \PY{p}{)}\PY{+w}{ }\PY{o}{\PYZhy{}\PYZgt{}}\PY{+w}{ }\PY{n}{jitter\PYZus{}steps\PYZus{}sleep\PYZus{}calories}
        \PY{n}{jitter\PYZus{}steps\PYZus{}sleep\PYZus{}calories}
    \end{Verbatim}
\end{tcolorbox}

\begin{Verbatim}[commandchars=\\\{\}]
    `geom\_smooth()` using method = 'loess' and formula = 'y \textasciitilde{} x'
    `geom\_smooth()` using method = 'loess' and formula = 'y \textasciitilde{} x'
\end{Verbatim}

\begin{center}
    \adjustimage{max size={0.9\linewidth}{0.9\paperheight}}{finalproject_files/finalproject_76_1.png}
\end{center}
{ \hspace*{\fill} \\}

\section{Conclusão}

Quais são algumas tendências no uso de dispositivos inteligentes?

\subsection{Achados}

\subsubsection{Qualidade de sono}

Com base em nossos resultados, podemos ver que os usuários dormem menos
de 8 horas por dia. Esse grupo de pessoas poderiam definir em nosso
aplicativo um horário desejado para dormir e receber uma notificação
minutos antes para se prepararem para dormir. O aplicativo poderia fazer
sugestões de atitudes a serem tomadas para que os usuários melhorem a
qualidade de seu sono. Por exemplo: Não comer muito (2-3 horas) antes de
antes de ir dormir (ref.
\href{https://health.clevelandclinic.org/is-eating-before-bed-bad-for-you}{1}),
evitar luzes fortes no período da noite especialemte luzes do teto e
celulares entre 22:00 e 04:00 (ref.
\href{https://youtu.be/h2aWYjSA1Jc?t=3733}{Sleep Toolkit: Tools for
    Optimizing Sleep \& Sleep-Wake Timing}); exercícios leves (caminhada,
ciclismo, corrida) ao ar livre no final da tarde pode ajudar a mitigar
os problema com o sono oriundos do uso de aparelhos eletrônicos na parte
da noite (ref. \href{https://youtu.be/nm1TxQj9IsQ?t=2614}{Master Your
    Sleep \& Be More Alert When Awake},
\href{https://youtu.be/h2aWYjSA1Jc?t=3858}{Sleep Toolkit: Tools for
    Optimizing Sleep \& Sleep-Wake Timing}), evitar o consumo de cafeina na
parte da tarde e noite (de 8 a 10 horas antes de dormir) (ref.
\href{https://youtu.be/h2aWYjSA1Jc?t=2272}{Sleep Toolkit: Tools for
    Optimizing Sleep \& Sleep-Wake Timing}), recomendar a pratica de
meditações que ajudem a relaxar e músicas para ajudar a cair no sono
(ref.
\href{https://hubermanlab.com/sleep-toolkit-tools-for-optimizing-sleep-and-sleep-wake-timing/}{Sleep
    Toolkit: Tools for Optimizing Sleep \& Sleep-Wake Timing},
\href{https://youtu.be/gbQFSMayJxk}{Dr.~Matthew Walker: The Science \&
    Practice of Perfecting Your Sleep}) e ver a luz do sol saindo de casa 30
a 60 minutos depois de acordar (ref.
\href{https://hubermanlab.com/toolkit-for-sleep/}{Toolkit for Sleep}).

\subsubsection{Calorias e Intensidade}

Apesar de trivial foi possível identificar, através de nossas análises,
que a \emph{Intensidade} da atividade do usuário refletiu na quantidade
de \emph{Calorias} queimadas. Existindo uma forte correlação entre as
duas variáveis (r = 0.9).

\subsubsection{Tempo de sono e passos no dia}

Quebrando as espectativas a quantidade de sono não afetou a quantidade
de passos no dia desse grupo de usuários. É Necessário mais dados para
que possamos elaborar uma explicação para esse fenfenômeno. Pois esse
grupo de usuários podem fazer parte de uma categoria de pessoas que
andar é parte crucial de sua rotina. Portanto, teriamos dados enviesados
por estarmos ``olhando'' para apenas uma parcela de nossos clientes.

\subsection{Recomendações}

\begin{enumerate}
    \def\labelenumi{\arabic{enumi}.}
    \tightlist
    \item
          Coletar mais dados
\end{enumerate}

O conjunto de dados baseado nos rastreadores pessoais Fitbit fornece
informações úteis sobre as preferências do usuário com as quais a
empresa pode aprender. Entretanto, o conjunto de dados fornece uma baixa
variação de usuários para que possamos entender o perfil dos nossos
clientes, contendo, apenas um conjunto de 33 usuários. Portanto, os
conjuntos de dados usados têm uma amostra pequena de dados e podem ser
tendenciosos, pois não tínhamos nenhum detalhe demográfico dos usuários,
e/ou dados categóricos. O nosso time de Dados recomenda a obtenção de
mais dados para que possamos fundar nossas hypothesis com mais robustez.

Como fazer: Seguindo as recomendações da
\href{https://www.bndes.gov.br/wps/portal/site/home/transparencia/lgpd}{Lei
    Geral de Proteção de Dados (LGPD)}.


Podemos fazer esse processo de obtenção de mais dados pela inclusão de descontos para a compra adicional ou recompensa ao usuário por meio do aplicativo Fitbit por um tempo de uso mais longo para aqueles que aceitarem preencher os formulários e/ou conceder acesso aos seus dados de uso. Entre essas formas podemos, também, fazer a coleta dos cookies com o estilo de navegação dos nossos usuários.
É de extremamente importância estarmos seguindo as recomendações de regras sobre o uso e concessão de dados pessoais presentes na \href{https://www.bndes.gov.br/wps/portal/site/home/transparencia/lgpd}{LGPD}.

\section{Código Fonte}

Todo código fonte desse projeto está presente nos seguintes repositórios públicos: \href{https://github.com/ARRETdaniel/22-2E_topicos_Especiais_em_IA_II_Sistemas_Inteligentes/tree/master/projectFinal}{GitHub}, \href{https://www.kaggle.com/code/arretdaniel/finalproject-fitbit-fitness-data}{Kaggle} sob a licença  \href{https://www.apache.org/licenses/LICENSE-2.0}{APACHE} versão 2.0.
% Add a bibliography block to the postdoc

% ---

% ----------------------------------------------------------
% PARTE
% ----------------------------------------------------------
%\part{Referenciais teóricos}
% ----------------------------------------------------------

% ---
% Capitulo de revisão de literatura

% ---



% ----------------------------------------------------------
% PARTE
% ----------------------------------------------------------
%\part{Resultados}
% ----------------------------------------------------------

% ---
% primeiro capitulo de Resultados
% ---

% ---

% ---


% ---
% segundo capitulo de Resultados
% ---


% ----------------------------------------------------------
% Finaliza a parte no bookmark do PDF
% para que se inicie o bookmark na raiz
% e adiciona espaço de parte no Sumário
% ----------------------------------------------------------
\phantompart

% ---
% Conclusão
% ---
%\chapter{Conclusão}
% ---


% ----------------------------------------------------------
% ELEMENTOS PÓS-TEXTUAIS
% ----------------------------------------------------------
\postextual
% ----------------------------------------------------------

% ----------------------------------------------------------
% Referências bibliográficas
% ----------------------------------------------------------
\bibliography{abntex2-modelo-references}



% ----------------------------------------------------------
% Glossário
% ----------------------------------------------------------
%
% Consulte o manual da classe abntex2 para orientações sobre o glossário.
%
%\glossary

% ----------------------------------------------------------
% Apêndices
% ----------------------------------------------------------

% ---
% Inicia os apêndices
\begin{comment}
% ---
\begin{apendicesenv}

	% Imprime uma página indicando o início dos apêndices
	\partapendices



	% ----------------------------------------------------------
	%\chapter{Quisque libero justo}
	% ----------------------------------------------------------

	\lipsum[50]

	% ----------------------------------------------------------
	\chapter{Nullam elementum urna vel imperdiet sodales elit ipsum pharetra ligula
	  ac pretium ante justo a nulla curabitur tristique arcu eu metus}
	% ----------------------------------------------------------
	\lipsum[55-57]

\end{apendicesenv}
% ---


% ----------------------------------------------------------
% Anexos
% ----------------------------------------------------------

% ---
% Inicia os anexos
% ---
\begin{anexosenv}

	% Imprime uma página indicando o início dos anexos
	\partanexos

	% ---
	%\chapter{Morbi ultrices rutrum lorem.}
	% ---
	\lipsum[30]

	% ---
	%\chapter{Cras non urna sed feugiat cum sociis natoque penatibus et magnis dis
	%parturient montes nascetur ridiculus mus}
	% ---

	\lipsum[31]

	% ---
	%\chapter{Fusce facilisis lacinia dui}
	% ---

	\lipsum[32]

\end{anexosenv}

\end{comment}
%---------------------------------------------------------------------
% INDICE REMISSIVO
%---------------------------------------------------------------------
\phantompart
\printindex
%---------------------------------------------------------------------

\end{document}
